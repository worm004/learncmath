\documentclass{article}
\usepackage[a4paper,hmargin=1.25in,vmargin=1in]{geometry}

\usepackage{listings} \usepackage[usenames,dvipsnames]{color}
\usepackage{amsmath}
\usepackage{latexsym}

\definecolor{DarkGreen}{rgb}{0.0,0.4,0.0}
\definecolor{highlight}{RGB}{255,251,204}

\lstdefinestyle{Style1}{
language=CPP, 
backgroundcolor=\color{highlight}, 
basicstyle=\footnotesize\ttfamily,
breakatwhitespace=false,
breaklines=true,
captionpos=b,
commentstyle=\usefont{T1}{pcr}{m}{sl}\color{DarkGreen},
deletekeywords={},
%escapeinside={\%}, % This allows you to escape to LaTeX using the character in the bracket
firstnumber=1,
frame=single,
frameround=tttt,
keywordstyle=\color{Blue}\bf,
morekeywords={},
numbers=left,
numbersep=10pt,
numberstyle=\tiny\color{Gray},
rulecolor=\color{black},
showstringspaces=false,
showtabs=false,
stepnumber=1,
stringstyle=\color{Purple},
tabsize=2,
}

\newcommand{\insertcode}[2]{\begin{itemize}\item[]\lstinputlisting[caption=#2,label=#1,style=Style1]{#1}\end{itemize}}

\begin{document}

\setcounter{section}{1}
\section{Sums}
\subsection{NOTATION}
$a_1 + ... + a_n$ could be presented as:
\begin{align}
\sum_{k=1}^{n}{a_k} = \sum_{k=0}^{n-1}{a_{k+1}} = \sum_{1\le k \le n}{a_k} = \sum_{1\le k+1 \le n}{a_{k+1}}.
\end{align}
Indicator is also useful.
\begin{align}
\sum_{k=1}^{n}{a_k} = \sum_{k}{a_k}[1\le k \le n].
\end{align}
The indicator is \textbf{harder} than others.
\begin{align}
\sum_{p}[p\le N]/p.
\end{align}
$p$ could be $0$ and the term $[0\le N]/0$ is 0.

\subsection{SUMS AND RECURRENCES}
\subsubsection{Simple Cases}
$S_n = \sum_{k=0}^n{a_k}$ can be converted into a recurrence problem:
\begin{align}
S_0 &= a_0;\\
S_n &= S_{n-1} + a_n. && \{n>0\}
\end{align}
Conversely, some recurrences can be reduced to sums.
\begin{align}
T_0 &= 0;\\
T_n &= 2T_{n-1}+1.&& \{n>0\}
\end{align}
Let $S_n =  T_n / (2n)$:
\begin{align}
S_0 &= 0;\\
S_n &= S_{n-1} + 2^{-n}. && \{n>0\}
\end{align}
Then
\begin{align}
S_n = \sum_{k=1}^{n}{2^{-k}}.
\end{align}

\subsubsection{A General Case}
The general form is:
\begin{align}
a_nT_n = b_nT_{n-1}+c_n.
\end{align}
Let $s_nb_n = s_{n-1}a_{n-1}$:
\begin{align}
s_na_nT_n =s_nb_nT_{n-1}+s_nc_n.
\end{align}
Then let $S_n = s_na_nT_n$:
\begin{align}
S_n &= S_{n-1}+s_nc_n;\\
S_n &= s_0a_0T_0 + \sum_{i=1}^{n}{s_ic_i};\\
S_n &= s_1b_1T_0 + \sum_{i=1}^{n}{s_ic_i}.
\end{align}
$T_n$ is solved:
\begin{align}
T_n = \frac{1}{s_na_n}(s_1b_1T_0 + \sum_{k=1}^{n}{s_kc_k}).
\end{align}
$s_n$ is:
\begin{align}
s_n = \frac{a_1...a_{n-1}}{b_2...b_{n}}.
\end{align}

\subsubsection{A Quick Sort Case}
\begin{align}
C_0 &= 0;\\
C_n &= n+1+\frac{2}{n}\sum_{k=0}^{n-1}{C_k}. && \{n>0\}
\end{align}
Multiply n on both side:
\begin{align}
nC_n &= n^2+n+2\sum_{k=0}^{n-1}{C_k}. && \{n>0\}\\
(n-1)C_{n-1} &= (n-1)^2+(n-1)+2\sum_{k=0}^{n-2}{C_k}. && \{n-1>0\}
\end{align}
Then
\begin{align}
C_0 &= 0;\\
nC_n &= (n+1)C_{n-1} + 2n. && \{n>0\}
\end{align}
And
\begin{align}
C_n = 2(n+1)\sum_{k=1}^{n}{\frac{1}{k+1}}.
\end{align}
Consider the harmonic number $H_n$.
\begin{align}
H_n = \sum_{k=1}^{n}{\frac{1}{k}}.
\end{align}
So
\begin{align}
C_n = 2(n+1)H_n - 2n.
\end{align}

\subsection{MANIPULATION OF SUMS}
\subsubsection{Basic Rules}

\begin{align}
\sum_{k\in K} ca_k &= c\sum_{k\in K}a_k; && \text{(Distributive law)}\\
\sum_{k\in K} {(a_k+b_k)} &= \sum_{k\in K}a_k + \sum_{k\in K}b_k; && \text{(Associative law)}\\
\sum_{k\in K} a_k &= \sum_{p(k)\in K}a_{p(k)}. && \text{(Commutative law)}
\end{align}
where $p(k)$ is some permutation.


Rule one.
\begin{align}
S_n &= \sum_{0\le k \le n}{(a+bk)} = \sum_{0\le n - k \le n}{(a+b(n-k))}.\\
2S_n &= \sum_{0 \le k \le n} {(2a+bn)} = (2a+bn) \sum_{0 \le k \le n} {1} = (2a+bn)(n+1).
\end{align}
Rule two.
\begin{align}
\sum_{k\in K}{a_k} + \sum_{k\in K^\prime}{a_k} = \sum_{k\in K \cap K^\prime}{a_k} + \sum_{k\in K \cup K^\prime}{a_k}.
\end{align}
Rule three.
\begin{align}
S_n + a_{n+1} = a_0 + \sum_{0\le k \le n}{a_{k+1}}.
\end{align}
Example one.
\begin{align}
S_n = \sum_{0\le k \le n}{ax^k}.
\end{align}
Use function 32.
\begin{align}
S_n + ax^{n+1} = ax^0 + \sum_{0\le k \le n}ax^{k+1} = ax^0 + xS_n.
\end{align}
Solution is:
\begin{align}
S_n &= \frac{a - ax^{n+1}}{1-x}; && \{1\neq x\}\\
S_n &= a(n+1). && \{\text{else}\}
\end{align}
Example two.
\begin{align}
S_n = \sum_{0\le k \le n}{k2^k}.
\end{align}
Use function 32.
\begin{align}
S_n + (n+1)2^{n+1} & = \sum_{0\le k \le n}(k+1)2^{k+1} \\
		   & = \sum_{0\le k \le n}k2^{k+1} + \sum_{0\le k \le n}2^{k+1}\\
		   & = 2S_n + 2^{n+2} - 2.
\end{align}
Solution is:
\begin{align}
S_n = (n-1)2^{n+1} + 2.
\end{align}
The general case.
\begin{align}
\sum_{0\le k \le n}{kx^k} = \frac{x - (n+1)x^{n+1}+nx^{n+2}}{(1-x)^2}. && \{x \neq 1\}
\end{align}

\subsection{MULTIPLE SUMS}
Notation:
\begin{align}
\sum_{1\le j,k \le 2}{a_j b_k} = a_1 b_1 + a_1 b_2 + a_2 b_1 + a_2 b_2.
\end{align}
Iverson's convention can also be applied in multiple sums.
\begin{align}
\sum_{P(j,k)}{a_{j,k}} = \sum_{j,k}{a_{j,k}[P(j,k)]}.
\end{align}
A sum of sums.
\begin{align}
\sum_j \sum_k{a_{j,k}[P(j,k)]} = \sum_j {\Big(\sum_k{a_{j,k}[P(j,k)]}\Big)}.
\end{align}
A law called interchanging the order of summation.
\begin{align}
\sum_j \sum_k{a_{j,k}[P(j,k)]} = \sum_{P(j,k)}{a_{j,k}} = \sum_k \sum_j{a_{j,k}[P(j,k)]}.
\end{align}
A general distributive law.
\begin{align}
\sum_{\substack{j\in J \\ k\in K}}{a_j b_k} = \Big( \sum_{j\in J}{a_j}\Big) \Big( \sum_{k\in K}{a_k} \Big).
\end{align}
Another way of the interchanging the order of summation law.
\begin{align}
\sum_{j\in J} \sum_{k \in K} {a_{j,k}} = \sum_{\substack{j\in J \\ k\in K}}{a_j b_k} = \sum_{k\in K} \sum_{j \in J} {a_{j,k}}.
\end{align}
When the range of an inner sum depends on the index variable of the outer sum, there is another way of the interchaning the order of summation law.
\begin{align}
\sum_{j\in J} \sum_{k\in K(j)} {a_{j,k}} = \sum_{k\in K^\prime} \sum_{j\in J^\prime(k)} {a_{j,k}}.
\end{align}
where
\begin{align}
[j\in J][k\in K(j)] = [k\in K^\prime][j\in J^\prime (k)].
\end{align}
Example one.
\begin{align}
[1 \le j \le n][j \le k \le n] = [1 \le j \le k \le n] = [1 \le k \le n][1 \le j \le k].
\end{align}
Furthermore:
\begin{align}
[1 \le j \le k \le n] + [1 \le k \le j \le n] = [1 \le k,j \le n] + [1 \le j = k \le n].
\end{align}
Example two.
\begin{align}
S = \sum_{1\le j < k \le n} {(a_k - a_j)(b_k - b_j)}.
\end{align}
Use the identity:
\begin{align}
[1 \le j < k \le n] + [1 \le k < j \le n] = [1 \le j,k \le n] - [1 \le j = k \le n]
\end{align}
Then
\begin{align}
2S &= \sum_{1\le j , k \le n} {(a_k - a_j)(b_k - b_j)} - 0 \\
   &= \sum_{1\le j , k \le n} {\big(a_k b_k + a_j b_j - a_k b_j - a_j b_k \big)} \\
   &= 2 \sum_{1\le j , k \le n} {a_j b_j} - 2 \sum_{1\le j , k \le n} {a_j b_k} \\
   &= 2n \sum_{1\le j \le n} {a_j b_j} - 2 \big( \sum_{1 \le j \le n} {a_j}\big) \big( \sum_{1 \le j \le n} {b_j} \big).\\
\end{align}
Solution is:
\begin{align}
 \sum_{1\le j < k \le n} {(a_k - a_j)(b_k - b_j)} &= n \sum_{1\le j \le n} {a_j b_j} -  \sum_{1 \le j \le n} {a_j} \sum_{1 \le j \le n} {b_j}.
\end{align}
This solution shows Chebyshev's monotonic inequalities:
\begin{align}
\big( \sum_{1 \le j \le n} {a_j} \big) \big( \sum_{1 \le j \le n} {b_j} \big) \le n \sum_{1\le j \le n} {a_j b_j}; && \{ \text{if } a_1 \le ... \le a_n \text{ and } b_1 \le ... \le b_n \} \\
\big( \sum_{1 \le j \le n} {a_j} \big) \big( \sum_{1 \le j \le n} {b_j} \big) \ge n \sum_{1\le j \le n} {a_j b_j}. && \{ \text{if } a_1 \le ... \le a_n \text{ and } b_1 \ge ... \ge b_n \}
\end{align}
One interesting formula.
\begin{align}
\sum_{0 \le k < n}{H_k} = nH_n - n.
\end{align}
\subsection{GENERAL METHODS}
Different methods can be used to solve:
\begin{align}
\Box_n = \sum_{0 \le k \le n} {k^2}.
\end{align}
Method 0: look it up.\\
Method 1: Guess a solution, prove it by induction.\\
Method 2: Perturb the sum.
\begin{align}
\sum_{0 \le k \le n} {k^3} + (n+1)^3 &= \sum_{0 \le k \le n+1} {k^3} = \sum_{0 \le k \le n} {(k+1)^3} \\
                                     &= \sum_{0 \le k \le n} (k^3 + 3k^2 + 3k + 1) \\
				     &= \sum_{0 \le k \le n} {k^3} + \sum_{0 \le k \le n} {(3k^2 + 3k + 1)};\\
(n+1)^3 &= \sum_{0 \le k \le n} {3k^2 + 3k + 1};\\
3\Box_n &= n (n+1) (n + \frac{1}{2}).
\end{align}
Method 3: Build a repertoire.
\begin{align}
R_0 &= \alpha; \\
R_n &= R_{n-1} + \beta + \gamma n + \sigma n^2;\\
R_n &= A(n)\alpha + B(n) \beta + C(n) \gamma + D(n) \sigma.
\end{align}
Let $R_n = n^3$ there is $\alpha = 0$, $\beta = 1$, $\gamma = -3$ and $\sigma = 3$.
\begin{align}
n^3 = 3D(n) - 3C(n) + B(n).
\end{align}
Let $R_n = \Box_n$ there is $\alpha = 0$, $\beta = 0$, $\gamma = 0$ and $\sigma = 1$.
\begin{align}
D(n) = \Box_n.
\end{align}
Let $R_n = n$ there is $\alpha = 0$, $\beta = 1$, $\gamma = 0$ and $\sigma = 0$.
\begin{align}
B(n) = n.
\end{align}
Let $R_n = n^2$ there is $\alpha = 0$, $\beta = -1$, $\gamma = 2$ and $\sigma = 0$.
\begin{align}
C(n) = \frac{n^2 + n}{2}.
\end{align}
Then 
\begin{align}
\Box_n = \frac {n^3 + 3C(n) - B(n)}{3}.
\end{align}
Method 4: Replace sums by integrals.
\begin{align}
E_n &= \Box_n - \int_{0}^{n}x^2dx = \Box_n - \frac{1}{3}x^3 = E_{n-1} + n - \frac{1}{3};\\
E_n &= \sum_{1 \le k \le n} (k - \frac{1}{3}).
\end{align}
Method 5: Expand and contract.
\begin{align}
\Box_n &= \sum_{1 \le k \le n} {k^2} \\
       &= \sum_{1 \le k \le n} \sum_{1 \le j \le k} k = \sum_{1 \le j \le n} \sum_{j \le k \le n} k \\
       &= \sum_{1 \le j \le n} \frac{n + j}{2}(n-j+1)\\
       &= \frac{1}{2}n(n+1)(n+\frac{1}{2}) - \frac{1}{2}\Box_n.
\end{align}
Method 6: Use finite calculus.\\
Method 7: Use generating functions.

\subsection{FINITE AND INFINITE CALCULUS}
Define $\bigtriangleup f(x) = f(x+1) - f(x)$, and
\begin{align}
x^{\underline {m}} &= x(x-1) ... (x-m+1); && \{m \ge 0\}\\
x^{\overline {m}} &= x(x+1) ... (x+m-1). && \{m \ge 0\}
\end{align}
when m is 0:
\begin{align}
x^{\underline 0} = x^{\overline 0} = 1.
\end{align}
This presentation is related to the factorial function.
\begin{align}
n! = n^{\underline n} = 1^{\overline n}.
\end{align}
Then
\begin{align}
\bigtriangleup (x^{\underline m}) = m x^{\underline {m-1}}.
\end{align}
The fundamental theorem of sum:
\begin{align}
g(x) = \bigtriangleup f(x). && \{\text{if and only if } \sum {g(x) \delta x} = f(x) + C\}
\end{align}
The finite sum:
\begin{align}
\sum\nolimits_{a}^{b}g(x)\delta x &= f(x)\big|_a^b = f(b) - f(a) && \{\text{if } g(x) = \bigtriangleup f(x)\}\\
                         &= \sum\nolimits_{a \le j < b} g(j).
\end{align}
Rule one.
\begin{align}
\sum\nolimits_{a}^{b}g(x)\delta x = -\sum\nolimits_{b}^{a}g(x)\delta x.
\end{align}
Rule two.
\begin{align}
\sum\nolimits_{a}^{b}g(x)\delta x +  \sum\nolimits_{b}^{c}g(x)\delta x = \sum\nolimits_{a}^{c}g(x)\delta x.
\end{align}
Sums of falling powers.
\begin{align}
\sum_{0 \le k < n} k^{\underline m} = \sum\nolimits_{0}^{n}k^{\underline m} = \frac{k^{\underline {m+1}}}{m+1}\Big|_0^n = \frac{n^{\underline {m+1}}}{m+1}. && \{\text{for } m \neq -1\}
\end{align}
Some examples.
\begin{align}
\sum_{0 \le k < n} k &= \sum_{0 \le k < n} k^{\underline 1} = \frac{n^{\underline 2}}{2};\\
\sum_{0 \le k < n} k^2 &= \sum_{0 \le k < n} (k^{\underline 2} + k^{\underline 1}) = \frac{n^{\underline 3}}{3} + \frac{n^{\underline 2}}{2};\\
\sum_{0 \le k < n} k^3 &= \sum_{0 \le k < n} (k^{\underline 3} + 3k^{\underline 2} + k^{\underline 1}) = \frac{n^{\underline 4}}{4} + n^{\underline 3} + \frac{n^{\underline 2}}{2}.
\end{align}
A negative rule.
\begin{align}
x^{\underline {-m}} = \frac{1}{(x+1) ... (x+m)}. &&\{\text{for } m>0\}
\end{align}
Another rule:
\begin{align}
x^{\underline {m+n}} = x^{\underline m} (x-m)^{\underline n}.
\end{align}
A complete description of the sums of falling powers.
\begin{align}
\sum\nolimits_{a}^{b}x^{\underline m}\delta x = 
\begin{cases}
\frac{k^{\underline {m+1}}}{m+1}\Big|_a^b; &\{\text{for } m \neq -1\}\\
H_x\big|_a^b.  &\{\text{for } m = -1\}
\end{cases}
\end{align}
Corresponding to $D(e^x) = e^x$:
\begin{align}
\bigtriangleup 2^x = 2^{x+1} - 2^x = 2^x.
\end{align}
One summary:
\\

\begin{tabular}{l l l l l}
\hline
$f = \sum g$ & $\bigtriangleup f = g$ & & $f = \sum g$ & $\bigtriangleup f = g$\\
\hline
$x^{\underline 0} = 1$ & $0$ & & $2^x$& $2^x$\\
$x^{\underline 1} = x$ & $1$ & & $c^x$& $(c-1)c^x$\\
$x^{\underline 2} = x(x-1)$ & $2x$ & & $c^x/(c-1)$& $c^x$\\
$x^{\underline m}$ & $mx^{\underline {m-1}}$ & & $cf$& $c\bigtriangleup f$\\
$x^{\underline {m+1}}/(m+1)$ & $x^{\underline m}$ & & $f+g$ & $\bigtriangleup f + \bigtriangleup g$\\
$H_x$ & $x^{\underline {-1}} = 1 / (x+1)$ & & $fg$& $f\bigtriangleup g + g\bigtriangleup f$\\
\hline
\end{tabular}
\\

$\bigtriangleup (u(x)v(x))$ does not have a nice form:
\begin{align}
\bigtriangleup(u(x)v(x)) &= u(x+1)v(x+1) - u(x)v(x) \\
			 &= u(x+1)v(x+1) - u(x)v(x+1) + u(x)v(x+1) - u(x)v(x)\\
			 &= u(x)\bigtriangleup v(x) + v(x+1)\bigtriangleup u(x).
\end{align}
Define 
\begin{align}
Ef(x) = f(x+1).
\end{align}
There is
\begin{align}
\bigtriangleup (uv) = u\bigtriangleup v + Ev\bigtriangleup u.
\end{align}
and 
\begin{align}
\sum u \bigtriangleup v = uv - \sum Ev \bigtriangleup u.
\end{align}
Example one.
\begin{align}
\sum x2^x \delta x = x2^x - \sum 2^{x+1} \delta x = x2^x - 2^{x+1} + C.
\end{align}
Example two.
\begin{align}
\sum xH_x \delta x &= \sum H_x \delta \frac{1}{2}x^{\underline 2} \\
		   &= \frac{x^{\underline 2}}{2}H_x - \sum \frac{1}{2}(x+1)^{\underline 2}\delta H_x \\
		   &= \frac{x^{\underline 2}}{2}H_x - \sum \frac{1}{2}(x+1)^{\underline 2}x^{\underline -1}\delta x \\
		   &= \frac{x^{\underline 2}}{2}H_x - \sum \frac{1}{2}x^{\underline 1} \delta x \\
		   &= \frac{x^{\underline 2}}{2}H_x - \frac{1}{4}x^{\underline 2} + C.
\end{align}

\subsection{INFINITE SUMS}
Let $x = x^+ - x^-$ where $x^+ = x[x > 0]$ and $x^- = -x[x<0]$.
The infiite sums can be presented as:
\begin{align}
\sum_{k \in K} a_k = \sum_{k \in K} a_k^+ - \sum_{k \in K} a_k^-.
\end{align}
Let $A^+ = \sum_{k \in K} a_k^+$ and $A^- = \sum_{k \in K} a_k^-$, $\sum_{k \in K}a_k$ is converge absolutely if both $A^+$ and $A^-$ is finite, $\sum_{k \in K}a_k$ is diverge to $\inf$ or $-\inf$ if $A^+$ is infinite or $A^-$ is infinite, else $\sum_{k \in K}a_k$ is undefined.


Many rules can be proved in converge absolutely cases.


The distributive law: if $\sum_{k \in K} a_k$ converges absolutely to $A$, then $\sum_{k \in K} ca_k$ converges absolutely to $cA$.


The associative law: if $\sum_{k \in K} a_k$ and $\sum_{k \in K} b_k$ converge absolutely to $A$ and $B$, then $\sum_{k \in K} (a _b + b_k)$ converges absolutely to $A+B$.


The commutative law: absolutely convergent sums over two or more indices can always be summed first with respect to any one of those indices.

\subsection{Exercises}
\textbf{Warmups 2.1:} 
\begin{align}
\sum_{k=4}^0 q_k = \sum_{k} q_k[4 \le k <= 0] = 0.
\end{align}

\noindent\rule{\textwidth}{0.4pt}
\textbf{Warmups 2.2:}
\begin{align}
x([x>0]-[x<0]) = |x|.
\end{align}

\noindent\rule{\textwidth}{0.4pt}
\textbf{Warmups 2.3:}
\begin{align}
\sum_{0 \le k \le 5} a_k &= a_0 + a_1 + a_2 + a_3 + a_4 + a_5;\\
\sum_{0 \le k^2 \le 5} a_{k^2} &= a_0 + a_1 + a_4.
\end{align}

\noindent\rule{\textwidth}{0.4pt}
\textbf{Warmups 2.4:}
\begin{align}
\sum_{1 \le i < j < k \le 4} a_{ijk} &= ((a_{123} + a_{124}) + a_{134}) + a_{234}; && \{\text{case a}\} \\
				     &= a_{123} + (a_{124} + (a_{134} + a_{234})). && \{\text{case b}\}
\end{align}

\noindent\rule{\textwidth}{0.4pt}
\textbf{Warmups 2.5:}
\begin{align}
(\sum_{j=1}^n a_j) (\sum_{k=1}^n \frac{1}{a_k}) &= \sum_{j=1}^n \sum_{k=1}^n \frac{a_j}{a_k}\\
						&\neq \sum_{k=1}^n \sum_{k=1}^n \frac{a_k}{a_k}.
\end{align}

\noindent\rule{\textwidth}{0.4pt}
\textbf{Warmups 2.6:}
\begin{align}
\sum_{k}[1\le j\le k \le n] &= 0; && \{\text{if } [1 \le j \le n]\}\\
			    &= n-j+1. && \{\text{else}\}
\end{align}

\noindent\rule{\textwidth}{0.4pt}
\textbf{Warmups 2.7:}
\begin{align}
\bigtriangledown (x^{\overline m}) = x^{\overline m} - (x-1)^{\overline m} = m x^{\overline {m-1}}.
\end{align}

\noindent\rule{\textwidth}{0.4pt}
\textbf{Warmups 2.8:}
\begin{align}
0^{\underline m} &= 0; && \{m > 0\}\\
		 &= \frac{1}{|m|!}. &&\{m \le 0\}
\end{align}


\noindent\rule{\textwidth}{0.4pt}
\textbf{Warmups 2.9:}
The definition is:
\begin{align}
x^{\overline m} &= x(x+1)...(x+m-1); && \{m > 0\}\\
		&= 1; &&\{m = 0\}\\
		&= \frac{1}{(x-1)...(x-m)}. &&\{m < 0\}
\end{align}
Based on the definition:
\begin{align}
x^{\overline {m+n}} = x^{\overline m}(x+m)^{\overline n}.
\end{align}
The solution is:
\begin{align}
x^{\overline {-n + n}} &= 1 = x^{\overline {-n}}(x-n)^{\overline n};\\
x^{\overline {-n}} &= \frac{1}{(x-n)^{\overline n}}.
\end{align}

\noindent\rule{\textwidth}{0.4pt}
\textbf{Warmups 2.10:}
\begin{align}
\bigtriangleup (uv) = u\bigtriangleup v + Ev\bigtriangleup u = v\bigtriangleup u + Eu\bigtriangleup v.
\end{align}

\noindent\rule{\textwidth}{0.4pt}
\textbf{Basics 2.11:}
\begin{align}
\sum_{0 \le k < n} (a_{k+1} - a_k) b_k &= \sum_{0 \le k < n} a_{k+1}b_k - \sum_{0 \le k < n} a_k b_k \\
				       &= \sum_{0 \le k < n} a_{k+1}b_k - (a_0 b_0 + \sum_{1 \le k \le n} a_k b_k - a_n b_n) \\
				       &= a_n b_n - a_0 b_0 - (\sum_{1 \le k \le n} a_k b_k - \sum_{0 \le k < n} a_{k+1}b_k) \\
				       &= a_n b_n - a_0 b_0 - \sum_{0 \le k < n} a_{k+1}(b_{k+1} - b_{k}).
\end{align}

\noindent\rule{\textwidth}{0.4pt}
\textbf{Basics 2.12:}
\begin{align}
p(k) &= k - c; && \{ k \text{ is odd} \} \\
     &= k + c. && \{ k \text{ is even} \}
\end{align}
Prove $p(k) \neq p(j)$ if $k \neq j$.
\begin{align}
k - c &\neq j - c; && \{\text{k and j is odd}\}\\
k + c &\neq j + c; && \{\text{k and j is even}\}\\
k - c &\neq j + c. && \{\text{k is odd and j is even}\}
\end{align}
Prove any integer number $n$ can be presented by $p(k)$.
\begin{align}
n = k - c;&& \{n+c\text{ is odd}\}\\
n = k + c.&& \{n-c\text{ is even}\}
\end{align}

\noindent\rule{\textwidth}{0.4pt}
\textbf{Basics 2.13:}
\begin{align}
f(0) &= \alpha;\\
f(n) &= f(n-1) + (-1)^n(\beta + n \gamma + n^2 \delta);\\
f(n) &= A(n) \alpha + B(n) \beta + C(n) \gamma + D(n) \delta.
\end{align}
Try
\begin{align}
f(n) &= 1;\\
f(n) &= (-1)^n;\\
f(n) &= n(-1)^n;\\
f(n) &= n^2(-1)^n.
\end{align}
Then
\begin{align}
f(n) = D(n) = \frac{1}{2}(-1)^n(n^2+n).
\end{align}

\noindent\rule{\textwidth}{0.4pt}
\textbf{Basics 2.14:}
\begin{align}
\sum_{k=1}^n k2^k &= \sum_{1\le j \le k \le n} 2^k = \sum_{1 \le j \le n} \sum_{j \le k \le n} 2^k \\
		  &= \sum_{1 \le j \le n} (2^{n+1} - 2^j) \\
		  &= (n-1)2^{n+1} + 2.
\end{align}

\noindent\rule{\textwidth}{0.4pt}
\textbf{Basics 2.15:}
\begin{align}
S3 &= \sum_{k=1}^nk^3;\\
S2 &= \sum_{k=1}^nk^2;\\
S1 &= \sum_{k=1}^nk.
\end{align}
Then
\begin{align}
S3 + S2 = 2\sum_{1 \le j \le k \le n}jk = (\sum_{k=1}^n k)^2 + \sum_{k=1}^n k^2 = (\sum_{k=1}^n k)^2 + S2.
\end{align}
Solution is:
\begin{align}
S3 = (\sum_{k=1}^n k)^2 = (\frac{1}{2}n(n+1))^2.
\end{align}

\noindent\rule{\textwidth}{0.4pt}
\textbf{Basics 2.16:}
\begin{align}
x^{\underline {m+n}} = x^{\underline m}(x-m)^{\underline n} = x^{\underline n}(x-n)^{\underline m}.
\end{align}

\noindent\rule{\textwidth}{0.4pt}
\textbf{Basics 2.17:}
Following rules and basic definitions can be used to prove states.
\begin{align}
(x+m-1)^{\underline m} (x-1)^{\underline {-m}} &= (x-1)^{\underline 0} = 1.\\
(x-m+1)^{\overline m} (x+1)^{\overline {-m}} &= (x+1)^{\overline 0} = 1.
\end{align}
%\begin{align}
%x^{\overline m} &= x(x+1)...(x+m-1) && \{m > 0\} \\
%		&= (-1)^m (-x)(-x-1)...(-x-m+1) = (-1)^m(-x)^{\underline m}\\
%		&= (x+m-1)...(x+1)x = (x+m-1)^{\underline m};\\
%x^{\overline m} &= 1 &&\{m = 0\}\\
%		&= (-1)^m(-x)^{\underline m} = (x+m-1)^{\underline m};\\
%x^{\overline m} &= \frac{1}{(x-1)...(x-m+1)} && \{m < 0\} \\
%		&= (-1)^m \frac{1}{(-x+1)...(-x+m-1)} = (-1)^m(-x)^{\underline m} \\
%		&= \frac{1}{(x-m+1)...(x-1)} = (x+m-1)^{\underline m}.\\
%\end{align}



\noindent\rule{\textwidth}{0.4pt}
\textbf{Basics 2.18:}
Unknown.

\noindent\rule{\textwidth}{0.4pt}
\textbf{Homework exercises 2.19:}
\begin{align}
T_0 &= 5;\\
2T_n &= nT_{n-1} + 3n! && \{ n > 0\}
\end{align}
Let $s_n = \frac{2^{n-1}}{n!}$ and mutiply to both sides:
\begin{align}
\frac{2^nT_n}{n!} = \frac{2^{n-1}T_{n-1}}{(n-1)!} + 3 *2^{n-1}.
\end{align}
Then
\begin{align}
S_0 &= 5; \\
S_n &= S_{n-1} + 3 * 2^{n-1}. 
\end{align}
And
\begin{align}
S_n = S_0 + 3(2^0 + ... + 2^{n-1}) = 5 + 3(2^n - 1) = 3*2^n + 2.
\end{align}
The solution is:
\begin{align}
T_0 &= 5;\\
T_n &= n!(3 + 2^{1-n}).
\end{align}

\noindent\rule{\textwidth}{0.4pt}
\textbf{Homework exercises 2.20:}
\begin{align}
\sum_{k=0}^n kH_k + (n+1)H_{n+1} &= \sum_{k=0}^n (k+1)H_{k+1} \\
				 &= \sum_{k=0}^n kH_k + \sum_{k=0}^n H_k + n.\\
\end{align}
Then 
\begin{align}
\sum_{k=0}^n H_k = (n+1)H_{n+1} - n - 1.
\end{align}

\noindent\rule{\textwidth}{0.4pt}
\textbf{Homework exercises 2.21:}
Rewrite formulas.
\begin{align}
S_n &= \sum_{k=0}^n (-1)^{n-k} = \sum_{k=0}^n (-1)^k; \\
T_n &= \sum_{k=0}^n (-1)^{n-k}k = \sum_{k=0}^n (-1)^{k}(n-k); \\
U_n &= \sum_{k=0}^n (-1)^{n-k}k^2 = \sum_{k=0}^n (-1)^{k}(n-k)^2.
\end{align}
Problem 1.
\begin{align}
S_n + (-1)^{n+1} = 1 + \sum_{k=0}^n (-1)^{k+1} = 1 - \sum_{k=0}^n (-1)^{k} = 1 - S_n.
\end{align}
Soluiton is
\begin{align}
S_n = \frac{1 + (-1)^n}{2} = [\text{n is even}].
\end{align}
Problem 2.
\begin{align}
T_{n+1} = \sum_{k=0}^{n+1}(-1)^k(n+1-k) = \sum_{k=0}^n(-1)^k(n+1-k) = T_n + S_n.
\end{align}
Then
\begin{align}
T_{n+1} = \sum_{k=0}^{n}S_k = \sum_{k=0}^{n}\frac{1+(-1)^k}{2} = \frac{n+1}{2} + \frac{S_n}{2}.
\end{align}
Solution is
\begin{align}
T_n = \frac{n + S_{n-1}}{2} = \frac{n + [\text{n is odd}]}{2}.
\end{align}
Problem 3.
\begin{align}
U_{n+1} &= \sum_{k=0}^{n+1}(-1)^k(n+1-k)^2 \\
	&= \sum_{k=0}^n(-1)^k((n-k)^2 + 2(n-k) + 1) \\
	&= U_n + S_n + 2T_n \\
	&= U_n + n + ([\text{n is odd}] + [\text{n is even}])\\
	&= U_n + n + 1.
\end{align}
Then 
\begin{align}
U_n = \frac{n(n+1)}{2}.
\end{align}

\noindent\rule{\textwidth}{0.4pt}
\textbf{Homework exercises 2.22:}
\begin{align}
\sum_{1 \le j < k \le n} (a_j b_k - a_k b_j)(A_j B_k - A_k B_j) &= \frac{1}{2} \sum_{1 \le j\text{, } k \le n} (a_j b_k - a_k b_j)(A_j B_k - A_k B_j) \\
								&= \sum_{1 \le j \le n} a_j A_j \sum_{1 \le j \le n} b_j B_j - \sum_{1 \le j \le n} b_j A_j \sum_{1 \le j \le n} a_j B_j.
\end{align}

\noindent\rule{\textwidth}{0.4pt}
\textbf{Homework exercises 2.23:}
Method a.
\begin{align}
\sum_{k=1}^n \frac{2k+1}{k(k+1)} &= \sum_{k=1}^n (2k+1)(\frac{1}{k} - \frac{1}{k+1}) \\
				 &= \sum_{k=1}^n (\frac{1}{k} + \frac{1}{k+1}) \\
				 &= 2H_n + \frac{1}{n+1} - 1.
\end{align}
Method b.
\begin{align}
\sum_{k=1}^n \frac{2k+1}{k(k+1)} &= \sum\nolimits_{1}^{n+1} \frac{2x+1}{x(x+1)} \delta x \\
				 &= \sum\nolimits_{1}^{n+1} -(2x+1) \delta (x-1)^{\underline {-1}} \\
				 &= -(2k+1)k^{\underline {-1}}\Big|_1^{n+1} + \sum\nolimits_{1}^{n+1} k^{\underline {-1}} \delta 2k\\
				 &= -1 - \frac{1}{n+1} + 2H_{n+1}.
\end{align}

\noindent\rule{\textwidth}{0.4pt}
\textbf{Homework exercises 2.24:}
\begin{align}
\sum\nolimits_{0}{n} \frac{H_x}{(x+1)(x+2)} \delta x &= \sum\nolimits_{0}{n} H_x x^{\underline {-2}} \delta x \\
						       &= \sum\nolimits_{0}{n} -H_x \delta x^{\underline {-1}} \\
						       &= -H_x x^{\underline {-1}}\Big|_0^{n} - \sum\nolimits_{0}^{n} -x^{\underline {-1}} \delta H_x \\
						       &= -H_x x^{\underline {-1}}\Big|_0^{n+1} + \sum\nolimits_{0}^{n} x^{\underline {-2}} \delta x\\
						       &= x^{\underline {-1}} (-1 - H_x) \Big|_0^{n+1}\\
						       &= \frac{n - H_n}{n+1}.
\end{align}
\noindent\rule{\textwidth}{0.4pt}
\textbf{Homework exercises 2.25:}
\begin{align}
\prod_{k \in K}a_k^c &= (\prod_{k \in K}a_k)^c; \\
\prod_{k \in K}a_k b_k &= \prod_{k \in K}a_k \prod_{k \in K}b_k; \\
\prod_{k \in K}a_k &= \prod_{p(k) \in K}a_{p(k)}; \\
\prod_{j \in J,k\in K} a_{j,k} &= \prod_{j \in J} \prod_{k \in K} a_{j,k}; \\
\prod_{k \in K}a_k &= \prod_k a_k^{[k \in K]};\\
\prod_{k \in K} c &= c^{\#K}.
\end{align}

\noindent\rule{\textwidth}{0.4pt}
\textbf{Homework exercises 2.26:}
\begin{align}
\prod_{1 \le j \le k \le n} &= \sqrt{\prod_{1 \le j,k\le n}a_j a_k \prod_{1 \le j = k \le n}a_j a_k} \\
			    &= \sqrt{\prod_{1 \le k \le n} a_k^{2n+2}} \\
			    &= \prod_{1 \le k \le n} a_k^{n+1}.
\end{align}

\noindent\rule{\textwidth}{0.4pt}
\textbf{Homework exercises 2.27:}
\begin{align}
\bigtriangleup (c^{\underline x}) = c^{\underline {x+1}} - c^{\underline {x}} =  c^{\underline {x}}(c - x - 1) = \frac{c^{\underline {x+2}}}{c-x}.
\end{align}
Then
\begin{align}
\sum_{k=1}^n \frac{(-2)^{\underline k}}{k} &= \sum\nolimits_1^{n+1} \frac{(-2)^{\underline x}}{x} \delta x \\
					   &= \sum\nolimits_1^{n+1} \delta -(-2)^{\underline {x-2}} \\
					   &= -(-2)^{\underline {x-2}} \Big|_1^{n+1} \\
					   &= -(-2)^{\underline {n-1}} + (-2) ^{\underline {-1}} \\
					   &= -(-1)^{n-1} n! - 1.
\end{align}

\noindent\rule{\textwidth}{0.4pt}
\textbf{Homework exercises 2.28:}
\begin{align}
\sum_{k\ge 1} \sum_{j \ge 1} (\frac{k}{j}[j = k+1] - \frac{j}{k}[j = k - 1]) \neq \sum_{j \ge 1} \sum_{k \ge 1} (\frac{k}{j}[j = k+1] - \frac{j}{k}[j = k - 1]).
\end{align}
Because the function is not converge absolutely, so the exchange of the two sum cannot be applied.

\noindent\rule{\textwidth}{0.4pt}
\textbf{Exam problems 2.29:}
\begin{align}
\sum_{k=1}^n \frac{(-1)^k k}{4k^2 - 1} &= \sum_{k=1}^n \frac{(-1)^k k}{(2k+1)(2k-1)} \\
				       &=  \frac{1}{2}\sum_{k=1}^n \frac{(-1)^k}{4k-2} + \frac{1}{2}\sum_{k=1}^n \frac{(-1)^k}{4k+2} \\
				       &= \frac{1}{2}(-\frac{1}{2} + \frac{(-1)^n}{4n+2}).
\end{align}

\noindent\rule{\textwidth}{0.4pt}
\textbf{Exam problems 2.30:}
\begin{align}
\sum_{x = a}^{b-1} x = \sum \nolimits_a^b x \delta x = \frac{1}{2}x^{\underline 2} \Big|_a^b  = \frac{1}{2}(b-a)(b+a-1).
\end{align}
Then 
\begin{align}
(b-a)(b+a-1) = 2100 = 2^2 * 3 * 5^2 * 7 = 2^{p_2} * 3^{p_3} * 5^{p_5}* 7^{p_7}.
\end{align}
Beacause $(b-a)$ and $b+a-1$ is a even number and a odd number.
The number of possible odd number could be 
\begin{align}
\prod_{k>2}(p_k + 1) = (1+1)(2+1)(1+1) = 12.
\end{align}

\noindent\rule{\textwidth}{0.4pt}
\textbf{Exam problems 2.31:}
a.
\begin{align}
\sum_{k \ge 2} (\zeta(k)-1) &= \sum_{k \ge 2} \sum_{j \ge 2} \frac{1}{j^k} \\
			    &= \sum_{j \ge 2} \sum_{k \ge 2} \frac{1}{j^k}  \\
			    &= \sum_{j \ge 2} \frac{\frac{1}{j^2}}{1 - \frac{1}{j}} \\
			    &= \sum_{j \ge 2} (\frac{1}{j-1} - \frac{1}{j}) \\
			    &= 1.
\end{align}
b.
\begin{align}
\sum_{k \ge 1} (\zeta(2k)-1) &= \sum_{k \ge 1} \sum_{j \ge 2} \frac{1}{j^{2k}} \\
			    &= \sum_{j \ge 2} \sum_{k \ge 2} \frac{1}{j^{2k}}  \\
			    &= \sum_{j \ge 2} \frac{1}{2}(\frac{1}{j-1} - \frac{1}{j+1}) \\
			    &= \frac{3}{4}.
\end{align}

\noindent\rule{\textwidth}{0.4pt}
\textbf{Exam problems 2.32:}
Let $S_0$ be the left function and $S_1$ be the right function.
When $2n \le x < 2n + 1$
\begin{align}
S_0 &= 1 + 2 + 3 + ... + n + (x-n-1) + ... + (x - 2n) \\
S_1 &= (x-1) + (x-3) + ... + (x - 2n + 1)
\end{align}
When $2n-1 \le x < 2n$
\begin{align}
S_0 &= 0 + 1 + 2 + 3 + ... + n-1 + (x-n) + ... + (x - 2n + 1) \\
S_1 &= (x-1) + (x-3) + ... + (x - 2n + 1)
\end{align}
Then solution is $n(x-n)$.

\noindent\rule{\textwidth}{0.4pt}
\textbf{Bonus problems 2.33:}
\begin{align}
\land_{k \in K}c a_k &= c\land_{k \in K} a_k; \\
\land_{k \in K} (a_k + b_k) &= \land_{k \in K}a_k + \land_{k \in K}b_k; \\
\land_{k \in K}a_k &= \land_{p(k) \in K}a_{p(k)}; \\
\land_{j \in J,k\in K} a_{j,k} &= \land_{j \in J} \land_{k \in K} a_{j,k}; \\
\land_{k \in K}a_k &= \land_k a_k \infty^{[k \not\in K]}.
\end{align}

\noindent\rule{\textwidth}{0.4pt}
\textbf{Bonus problems 2.34:}
This problem is not perfectly solved.


If the $\sum_{k \in K} a_k$ is undefined, $\sum_{k \in K} a_k^+$ and $\sum_{k \in K} a_k^-$ are all equal to $\infty$.
This means $\sum_{k \in K} a_k$ can be larger or smaller to any value.
Then there must exist a $E_1$:
\begin{align}
\sum_{k \in F_1} a_k = \sum_{k \in E_1} a_k \le A^-.
\end{align}
For the rest $F_n$, let
\begin{align}
F_n = F_{n-1} \cup E_n.
\end{align}
When n is even:
\begin{align}
\sum_{k \in F_n} a_k = \sum_{k \in F_{n-1} \cup E_n} a_k = \sum_{k \in F_{n-1}} a_k + \sum_{k \in E_n} a_k - \sum_{k \in F_{n-1} \cap E_n} a_k \ge A^+.
\end{align}
Then there must exist a $E_n$ that $F_{n-1} \cap E_n \neq \phi$:
\begin{align}
\sum_{k \in E_n} a_k \ge A^+ - \sum_{k \in F_{n-1}} a_k + \sum_{k \in F_{n-1} \cap E_n} a_k.
\end{align}
When n is odd:
\begin{align}
\sum_{k \in F_n} a_k = \sum_{k \in F_{n-1} \cup E_n} a_k = \sum_{k \in F_{n-1}} a_k + \sum_{k \in E_n} a_k - \sum_{k \in F_{n-1} \cap E_n} a_k \le A^-.
\end{align}
Then there must exist a $E_n$ that $F_{n-1} \cap E_n \neq \phi$:
\begin{align}
\sum_{k \in E_n} a_k \le A^- - \sum_{k \in F_{n-1}} a_k + \sum_{k \in F_{n-1} \cap E_n} a_k.
\end{align}

\noindent\rule{\textwidth}{0.4pt}
\textbf{Bonus problems 2.35:}
Perfect power: $n$ is a perfect power if there exist natural numbers $m > 1$, and $k > 1$ such that $m^k = n$.


Goldbach Euler theorem:
\begin{align}
\sum_{k\in P} \frac{1}{k-1} = 1
\end{align}
Let P is the perfect power set and T is the nopower set.
\begin{align}
\sum_{k\in P} \frac{1}{k-1} &= \sum_{k\in P} (k-1)^{-1} \\
			    &= \sum_{i \ge 2} \sum_{a \in T}(a^i-1)^{-1} \\
			    &= \sum_{i \ge 2} \sum_{a \in T} \sum_{j \ge 1} a^{-ij} \\
			    &= \sum_{n \ge 2} \sum_{k \ge 2} n^{-k} \\
			    &= \sum_{n \ge 2} (n(n-1))^{-1} = 1.
\end{align}

\noindent\rule{\textwidth}{0.4pt}
\textbf{Bonus problems 2.36:}
a. follow the definition:
\begin{align}
g(1) &= 1; \\
g(n) - g(n-1) &= f(n). && \{n > 1\}
\end{align}
b. according to the definition, $f(g(k)) = n$ when $k \in (g(n-1), g(n)]$.
\begin{align}
g(g(n)) - g(g(n-1)) &= \sum_{i = 1}^g(n) f(i) - \sum_{i = 1}^{g(n-1)} f(i) \\
		    &= \sum_{i} f(i)[g(n-1) < i \le g(n)] \\
		    &= n f(n).
\end{align}
c. 
\begin{align}
g(g(g(n))) - g(g(g(n-1))) &= \sum_{k=1}^{g(n)} kf(k) - \sum_{k=1}^{g(n-1)} kf(k) \\
			  &= \sum_k kf(k)[g(n-1) < k \le g(n)] \\
			  &= n \sum_{k = g(n-1) + 1}^{g(n)}k.
\end{align}

\noindent\rule{\textwidth}{0.4pt}
\textbf{Research problem 2.37:}
It seems that a book named "Research Problems in Discrete Geometry" discussed this problem in chapter 3.
However I cannot got a copy of the book.
\end{document}
