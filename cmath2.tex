\documentclass{article}
\usepackage[a4paper,hmargin=1.25in,vmargin=1in]{geometry}

\usepackage{listings}
\usepackage[usenames,dvipsnames]{color}
\usepackage{amsmath}

\definecolor{DarkGreen}{rgb}{0.0,0.4,0.0}
\definecolor{highlight}{RGB}{255,251,204}

\lstdefinestyle{Style1}{
language=CPP, 
backgroundcolor=\color{highlight}, 
basicstyle=\footnotesize\ttfamily,
breakatwhitespace=false,
breaklines=true,
captionpos=b,
commentstyle=\usefont{T1}{pcr}{m}{sl}\color{DarkGreen},
deletekeywords={},
%escapeinside={\%}, % This allows you to escape to LaTeX using the character in the bracket
firstnumber=1,
frame=single,
frameround=tttt,
keywordstyle=\color{Blue}\bf,
morekeywords={},
numbers=left,
numbersep=10pt,
numberstyle=\tiny\color{Gray},
rulecolor=\color{black},
showstringspaces=false,
showtabs=false,
stepnumber=1,
stringstyle=\color{Purple},
tabsize=2,
}

\newcommand{\insertcode}[2]{\begin{itemize}\item[]\lstinputlisting[caption=#2,label=#1,style=Style1]{#1}\end{itemize}}

\begin{document}

\section{Sums}
\subsection{NOTATION}
$a_1,...,a_n$ could be presented as:
\begin{align}
\sum_{k=1}^{n}{a_k} = \sum_{k=0}^{n-1}{a_{k+1}} = \sum_{1\le k \le n}{a_k} = \sum_{1\le k+1 \le n}{a_{k+1}}.
\end{align}
Indicator is also can be used:
\begin{align}
\sum_{k=1}^{n}{a_k} = \sum_{k}{a_k}[i\le k \le n].
\end{align}
The indicator could be more powerful than others:
\begin{align}
\sum_{p}[p\le N]/p.
\end{align}
p could be 0 and the term $[0\le N]/0$ is 0.

\subsection{SUMS AND RECURRENCES}
\subsubsection{Simple Cases}
One way to solve $S_n = \sum_{k=0}^n{a_k}$ is to convert the problem into a recurrence problem:
\begin{align}
S_0 &= a_0;\\
S_n &= S_{n-1} + a_n. && \{n>0\}
\end{align}
Conversely, some recurrences can be reduced to sums:
\begin{align}
T_0 &= 0;\\
T_n &= 2T_{n-1}+1.&& \{n>0\}
\end{align}
Set $S_n = \frac{T_n}{2n}$ it turns:
\begin{align}
S_0 &= 0;\\
S_n &= S_{n-1} + 2^{-n}. && \{n>0\}
\end{align}
which means $S_n = \sum_{k=1}^{n}{2^{-k}}$.

\subsubsection{A General Case}
A more general case is
\begin{align}
a_nT_n = b_nT_{n-1}+c_n.
\end{align}
and use a $s_n$ where $s_nb_n = s_{n-1}a_{n-1}$:
\begin{align}
s_na_nT_n =s_nb_nT_{n-1}+s_nc_n.
\end{align}
Let $S_n = s_na_nT_n$ there is
\begin{align}
S_n = S_{n-1}+s_nc_n.
\end{align}
and
\begin{align}
S_n = s_0a_0T_0 + \sum_{i=1}^{n}{s_ic_i} = s_1b_1T_0 + \sum_{i=1}^{n}{s_ic_i}.
\end{align}
Hence $T_n$ is solved:
\begin{align}
T_n = \frac{1}{s_na_n}(s_1b_1T_0 + \sum_{k=1}^{n}{s_kc_k}).
\end{align}
The choice of $s_n$ is:
\begin{align}
s_n = \frac{a_1...a_{n-1}}{b_2...b_{n}}.
\end{align}

\subsubsection{A Quick Sort Case}
\begin{align}
C_0 &= 0;\\
C_n &= n+1+\frac{2}{n}\sum_{k=0}^{n-1}{C_k}. && \{n>0\}
\end{align}
The solution is to multiply n on both side:
\begin{align}
nC_n &= n^2+n+2\sum_{k=0}^{n-1}{C_k}. && \{n>0\}\\
(n-1)C_{n-1} &= (n-1)^2+(n-1)+2\sum_{k=0}^{n-2}{C_k}. && \{n-1>0\}
\end{align}
then
\begin{align}
C_0 &= 0;\\
nC_n &= (n+1)C_{n-1} + 2n. && \{n>0\}
\end{align}
So the solution is
\begin{align}
C_n = 2(n+1)\sum_{k=1}^{n}{\frac{1}{k+1}}.
\end{align}
Harmonic $H_n$ is
\begin{align}
H_n = \sum_{k=1}^{n}{\frac{1}{k}}.
\end{align}
So $C_n$ can be presented in a shorter way:
\begin{align}
C_n = 2(n+1)H_n - 2n.
\end{align}

\subsection{MANIPULATION OF SUMS}
\subsubsection{Basic Rules}

\begin{align}
\sum_{k\in K} ca_k &= c\sum_{k\in K}a_k; && \text{Distributive law}\\
\sum_{k\in K} {(a_k+b_k)} &= \sum_{k\in K}a_k + \sum_{k\in K}b_k; && \text{Associative law}\\
\sum_{k\in K} a_k &= \sum_{p(k)\in K}a_{p(k)}. && \text{Commutative law}
\end{align}

%where $p(k)$ re-orders the terms.
%Then
%$$\sum_{k\in K} {a_k} + \sum_{k\in K'}{a_k} = \sum_{k\in K \cap K'} {a_k} + \sum_{k\in K' \cup K'}{a_k} $$
%
%\subsection{MULTIPLE SUMS}

%\insertcode{"scripts/init0.zsh"}{An easy init example}


\end{document}
