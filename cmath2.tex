\documentclass{article}
\usepackage[a4paper,hmargin=1.25in,vmargin=1in]{geometry}

\usepackage{listings}
\usepackage[usenames,dvipsnames]{color}
\usepackage{amsmath}
\usepackage{latexsym}

\definecolor{DarkGreen}{rgb}{0.0,0.4,0.0}
\definecolor{highlight}{RGB}{255,251,204}

\lstdefinestyle{Style1}{
language=CPP, 
backgroundcolor=\color{highlight}, 
basicstyle=\footnotesize\ttfamily,
breakatwhitespace=false,
breaklines=true,
captionpos=b,
commentstyle=\usefont{T1}{pcr}{m}{sl}\color{DarkGreen},
deletekeywords={},
%escapeinside={\%}, % This allows you to escape to LaTeX using the character in the bracket
firstnumber=1,
frame=single,
frameround=tttt,
keywordstyle=\color{Blue}\bf,
morekeywords={},
numbers=left,
numbersep=10pt,
numberstyle=\tiny\color{Gray},
rulecolor=\color{black},
showstringspaces=false,
showtabs=false,
stepnumber=1,
stringstyle=\color{Purple},
tabsize=2,
}

\newcommand{\insertcode}[2]{\begin{itemize}\item[]\lstinputlisting[caption=#2,label=#1,style=Style1]{#1}\end{itemize}}

\begin{document}

\section{Sums}
\subsection{NOTATION}
$a_1 + ... + a_n$ could be presented as:
\begin{align}
\sum_{k=1}^{n}{a_k} = \sum_{k=0}^{n-1}{a_{k+1}} = \sum_{1\le k \le n}{a_k} = \sum_{1\le k+1 \le n}{a_{k+1}}.
\end{align}
Indicator is also useful.
\begin{align}
\sum_{k=1}^{n}{a_k} = \sum_{k}{a_k}[1\le k \le n].
\end{align}
The indicator is \textbf{harder} than others.
\begin{align}
\sum_{p}[p\le N]/p.
\end{align}
$p$ could be $0$ and the term $[0\le N]/0$ is 0.

\subsection{SUMS AND RECURRENCES}
\subsubsection{Simple Cases}
$S_n = \sum_{k=0}^n{a_k}$ can be converted into a recurrence problem:
\begin{align}
S_0 &= a_0;\\
S_n &= S_{n-1} + a_n. && \{n>0\}
\end{align}
Conversely, some recurrences can be reduced to sums.
\begin{align}
T_0 &= 0;\\
T_n &= 2T_{n-1}+1.&& \{n>0\}
\end{align}
Let $S_n =  T_n / (2n)$:
\begin{align}
S_0 &= 0;\\
S_n &= S_{n-1} + 2^{-n}. && \{n>0\}
\end{align}
Then
\begin{align}
S_n = \sum_{k=1}^{n}{2^{-k}}.
\end{align}

\subsubsection{A General Case}
The general form is:
\begin{align}
a_nT_n = b_nT_{n-1}+c_n.
\end{align}
Let $s_nb_n = s_{n-1}a_{n-1}$:
\begin{align}
s_na_nT_n =s_nb_nT_{n-1}+s_nc_n.
\end{align}
Then let $S_n = s_na_nT_n$:
\begin{align}
S_n &= S_{n-1}+s_nc_n;\\
S_n &= s_0a_0T_0 + \sum_{i=1}^{n}{s_ic_i};\\
S_n &= s_1b_1T_0 + \sum_{i=1}^{n}{s_ic_i}.
\end{align}
$T_n$ is solved:
\begin{align}
T_n = \frac{1}{s_na_n}(s_1b_1T_0 + \sum_{k=1}^{n}{s_kc_k}).
\end{align}
$s_n$ is:
\begin{align}
s_n = \frac{a_1...a_{n-1}}{b_2...b_{n}}.
\end{align}

\subsubsection{A Quick Sort Case}
\begin{align}
C_0 &= 0;\\
C_n &= n+1+\frac{2}{n}\sum_{k=0}^{n-1}{C_k}. && \{n>0\}
\end{align}
Multiply n on both side:
\begin{align}
nC_n &= n^2+n+2\sum_{k=0}^{n-1}{C_k}. && \{n>0\}\\
(n-1)C_{n-1} &= (n-1)^2+(n-1)+2\sum_{k=0}^{n-2}{C_k}. && \{n-1>0\}
\end{align}
Then
\begin{align}
C_0 &= 0;\\
nC_n &= (n+1)C_{n-1} + 2n. && \{n>0\}
\end{align}
And
\begin{align}
C_n = 2(n+1)\sum_{k=1}^{n}{\frac{1}{k+1}}.
\end{align}
Consider the harmonic number $H_n$.
\begin{align}
H_n = \sum_{k=1}^{n}{\frac{1}{k}}.
\end{align}
So
\begin{align}
C_n = 2(n+1)H_n - 2n.
\end{align}

\subsection{MANIPULATION OF SUMS}
\subsubsection{Basic Rules}

\begin{align}
\sum_{k\in K} ca_k &= c\sum_{k\in K}a_k; && \text{(Distributive law)}\\
\sum_{k\in K} {(a_k+b_k)} &= \sum_{k\in K}a_k + \sum_{k\in K}b_k; && \text{(Associative law)}\\
\sum_{k\in K} a_k &= \sum_{p(k)\in K}a_{p(k)}. && \text{(Commutative law)}
\end{align}
where $p(k)$ is some permutation.


Rule one.
\begin{align}
S_n &= \sum_{0\le k \le n}{(a+bk)} = \sum_{0\le n - k \le n}{(a+b(n-k))}.\\
2S_n &= \sum_{0 \le k \le n} {(2a+bn)} = (2a+bn) \sum_{0 \le k \le n} {1} = (2a+bn)(n+1).
\end{align}
Rule two.
\begin{align}
\sum_{k\in K}{a_k} + \sum_{k\in K^\prime}{a_k} = \sum_{k\in K \cap K^\prime}{a_k} + \sum_{k\in K \cup K^\prime}{a_k}.
\end{align}
Rule three.
\begin{align}
S_n + a_{n+1} = a_0 + \sum_{0\le k \le n}{a_{k+1}}.
\end{align}
Example one.
\begin{align}
S_n = \sum_{0\le k \le n}{ax^k}.
\end{align}
Use function 32.
\begin{align}
S_n + ax^{n+1} = ax^0 + \sum_{0\le k \le n}ax^{k+1} = ax^0 + xS_n.
\end{align}
Solution is:
\begin{align}
S_n &= \frac{a - ax^{n+1}}{1-x}; && \{1\neq x\}\\
S_n &= a(n+1). && \{\text{else}\}
\end{align}
Example two.
\begin{align}
S_n = \sum_{0\le k \le n}{k2^k}.
\end{align}
Use function 32.
\begin{align}
S_n + (n+1)2^{n+1} & = \sum_{0\le k \le n}(k+1)2^{k+1} \\
		   & = \sum_{0\le k \le n}k2^{k+1} + \sum_{0\le k \le n}2^{k+1}\\
		   & = 2S_n + 2^{n+2} - 2.
\end{align}
Solution is:
\begin{align}
S_n = (n-1)2^{n+1} + 2.
\end{align}
The general case.
\begin{align}
\sum_{0\le k \le n}{kx^k} = \frac{x - (n+1)x^{n+1}+nx^{n+2}}{(1-x)^2}. && \{x \neq 1\}
\end{align}

\subsection{MULTIPLE SUMS}
Notation:
\begin{align}
\sum_{1\le j,k \le 2}{a_j b_k} = a_1 b_1 + a_1 b_2 + a_2 b_1 + a_2 b_2.
\end{align}
Iverson's convention can also be applied in multiple sums.
\begin{align}
\sum_{P(j,k)}{a_{j,k}} = \sum_{j,k}{a_{j,k}[P(j,k)]}.
\end{align}
A sum of sums.
\begin{align}
\sum_j \sum_k{a_{j,k}[P(j,k)]} = \sum_j {\Big(\sum_k{a_{j,k}[P(j,k)]}\Big)}.
\end{align}
A law called interchanging the order of summation.
\begin{align}
\sum_j \sum_k{a_{j,k}[P(j,k)]} = \sum_{P(j,k)}{a_{j,k}} = \sum_k \sum_j{a_{j,k}[P(j,k)]}.
\end{align}
A general distributive law.
\begin{align}
\sum_{\substack{j\in J \\ k\in K}}{a_j b_k} = \Big( \sum_{j\in J}{a_j}\Big) \Big( \sum_{k\in K}{a_k} \Big).
\end{align}
Another way of the interchanging the order of summation law.
\begin{align}
\sum_{j\in J} \sum_{k \in K} {a_{j,k}} = \sum_{\substack{j\in J \\ k\in K}}{a_j b_k} = \sum_{k\in K} \sum_{j \in J} {a_{j,k}}.
\end{align}
When the range of an inner sum depends on the index variable of the outer sum, there is another way of the interchaning the order of summation law.
\begin{align}
\sum_{j\in J} \sum_{k\in K(j)} {a_{j,k}} = \sum_{k\in K^\prime} \sum_{j\in J^\prime(k)} {a_{j,k}}.
\end{align}
where
\begin{align}
[j\in J][k\in K(j)] = [k\in K^\prime][j\in J^\prime (k)].
\end{align}
Example one.
\begin{align}
[1 \le j \le n][j \le k \le n] = [1 \le j \le k \le n] = [1 \le k \le n][1 \le j \le k].
\end{align}
Furthermore:
\begin{align}
[1 \le j \le k \le n] + [1 \le k \le j \le n] = [1 \le k,j \le n] + [1 \le j = k \le n].
\end{align}
Example two.
\begin{align}
S = \sum_{1\le j < k \le n} {(a_k - a_j)(b_k - b_j)}.
\end{align}
Use the identity:
\begin{align}
[1 \le j < k \le n] + [1 \le k < j \le n] = [1 \le j,k \le n] - [1 \le j = k \le n]
\end{align}
Then
\begin{align}
2S &= \sum_{1\le j , k \le n} {(a_k - a_j)(b_k - b_j)} - 0 \\
   &= \sum_{1\le j , k \le n} {\big(a_k b_k + a_j b_j - a_k b_j - a_j b_k \big)} \\
   &= 2 \sum_{1\le j , k \le n} {a_j b_j} - 2 \sum_{1\le j , k \le n} {a_j b_k} \\
   &= 2n \sum_{1\le j \le n} {a_j b_j} - 2 \big( \sum_{1 \le j \le n} {a_j}\big) \big( \sum_{1 \le j \le n} {b_j} \big).\\
\end{align}
Solution is:
\begin{align}
 \sum_{1\le j < k \le n} {(a_k - a_j)(b_k - b_j)} &= n \sum_{1\le j \le n} {a_j b_j} -  \sum_{1 \le j \le n} {a_j} \sum_{1 \le j \le n} {b_j}.
\end{align}
This solution shows Chebyshev's monotonic inequalities:
\begin{align}
\big( \sum_{1 \le j \le n} {a_j} \big) \big( \sum_{1 \le j \le n} {b_j} \big) \le n \sum_{1\le j \le n} {a_j b_j}; && \{ \text{if } a_1 \le ... \le a_n \text{ and } b_1 \le ... \le b_n \} \\
\big( \sum_{1 \le j \le n} {a_j} \big) \big( \sum_{1 \le j \le n} {b_j} \big) \ge n \sum_{1\le j \le n} {a_j b_j}. && \{ \text{if } a_1 \le ... \le a_n \text{ and } b_1 \ge ... \ge b_n \}
\end{align}
One interesting formula.
\begin{align}
\sum_{0 \le k < n}{H_k} = nH_n - n.
\end{align}
\subsection{GENERAL METHODS}
Different methods can be used to solve:
\begin{align}
\Box_n = \sum_{0 \le k \le n} {k^2}.
\end{align}
Method 0: look it up.\\
Method 1: Guess a solution, prove it by induction.\\
Method 2: Perturb the sum.
\begin{align}
\sum_{0 \le k \le n} {k^3} + (n+1)^3 &= \sum_{0 \le k \le n+1} {k^3} = \sum_{0 \le k \le n} {(k+1)^3} \\
                                     &= \sum_{0 \le k \le n} (k^3 + 3k^2 + 3k + 1) \\
				     &= \sum_{0 \le k \le n} {k^3} + \sum_{0 \le k \le n} {(3k^2 + 3k + 1)};\\
(n+1)^3 &= \sum_{0 \le k \le n} {3k^2 + 3k + 1};\\
3\Box_n &= n (n+1) (n + \frac{1}{2}).
\end{align}
Method 3: Build a repertoire.
\begin{align}
R_0 &= \alpha; \\
R_n &= R_{n-1} + \beta + \gamma n + \sigma n^2;\\
R_n &= A(n)\alpha + B(n) \beta + C(n) \gamma + D(n) \sigma.
\end{align}
Let $R_n = n^3$ there is $\alpha = 0$, $\beta = 1$, $\gamma = -3$ and $\sigma = 3$.
\begin{align}
n^3 = 3D(n) - 3C(n) + B(n).
\end{align}
Let $R_n = \Box_n$ there is $\alpha = 0$, $\beta = 0$, $\gamma = 0$ and $\sigma = 1$.
\begin{align}
D(n) = \Box_n.
\end{align}
Let $R_n = n$ there is $\alpha = 0$, $\beta = 1$, $\gamma = 0$ and $\sigma = 0$.
\begin{align}
B(n) = n.
\end{align}
Let $R_n = n^2$ there is $\alpha = 0$, $\beta = -1$, $\gamma = 2$ and $\sigma = 0$.
\begin{align}
C(n) = \frac{n^2 + n}{2}.
\end{align}
Then 
\begin{align}
\Box_n = \frac {n^3 + 3C(n) - B(n)}{3}.
\end{align}
Method 4: Replace sums by integrals.
\begin{align}
E_n &= \Box_n - \int_{0}^{n}x^2dx = \Box_n - \frac{1}{3}x^3 = E_{n-1} + n - \frac{1}{3};\\
E_n &= \sum_{1 \le k \le n} (k - \frac{1}{3}).
\end{align}
Method 5: Expand and contract.
\begin{align}
\Box_n &= \sum_{1 \le k \le n} {k^2} \\
       &= \sum_{1 \le k \le n} \sum_{1 \le j \le k} k = \sum_{1 \le j \le n} \sum_{j \le k \le n} k \\
       &= \sum_{1 \le j \le n} \frac{n + j}{2}(n-j+1)\\
       &= \frac{1}{2}n(n+1)(n+\frac{1}{2}) - \frac{1}{2}\Box_n.
\end{align}
Method 6: Use finite calculus.\\
Method 7: Use generating functions.

\subsection{FINITE AND INFINITE CALCULUS}
Define $\bigtriangleup f(x) = f(x+1) - f(x)$, and
\begin{align}
x^{\underline {m}} &= x(x-1) ... (x-m+1); && \{m \ge 0\}\\
x^{\overline {m}} &= x(x+1) ... (x+m-1). && \{m \ge 0\}
\end{align}
when m is 0:
\begin{align}
x^{\underline 0} = x^{\overline 0} = 1.
\end{align}
This presentation is related to the factorial function.
\begin{align}
n! = n^{\underline n} = 1^{\overline n}.
\end{align}
Then
\begin{align}
\bigtriangleup (x^{\underline m}) = m x^{\underline {m-1}}.
\end{align}
The fundamental theorem of sum:
\begin{align}
g(x) = \bigtriangleup f(x). && \{\text{if and only if } \sum {g(x) \delta x} = f(x) + C\}
\end{align}
The finite sum:
\begin{align}
\sum\nolimits_{a}^{b}g(x)\delta x &= f(x)\big|_a^b = f(b) - f(a) && \{\text{if } g(x) = \bigtriangleup f(x)\}\\
                         &= \sum\nolimits_{a \le j < b} g(j).
\end{align}
Rule one.
\begin{align}
\sum\nolimits_{a}^{b}g(x)\delta x = -\sum\nolimits_{b}^{a}g(x)\delta x.
\end{align}
Rule two.
\begin{align}
\sum\nolimits_{a}^{b}g(x)\delta x +  \sum\nolimits_{b}^{c}g(x)\delta x = \sum\nolimits_{a}^{c}g(x)\delta x.
\end{align}
Sums of falling powers.
\begin{align}
\sum_{0 \le k < n} k^{\underline m} = \sum\nolimits_{0}^{n}k^{\underline m} = \frac{k^{\underline {m+1}}}{m+1}\Big|_0^n = \frac{n^{\underline {m+1}}}{m+1}. && \{\text{for } m \neq -1\}
\end{align}
Some examples.
\begin{align}
\sum_{0 \le k < n} k &= \sum_{0 \le k < n} k^{\underline 1} = \frac{n^{\underline 2}}{2};\\
\sum_{0 \le k < n} k^2 &= \sum_{0 \le k < n} (k^{\underline 2} + k^{\underline 1}) = \frac{n^{\underline 3}}{3} + \frac{n^{\underline 2}}{2};\\
\sum_{0 \le k < n} k^3 &= \sum_{0 \le k < n} (k^{\underline 3} + 3k^{\underline 2} + k^{\underline 1}) = \frac{n^{\underline 4}}{4} + n^{\underline 3} + \frac{n^{\underline 2}}{2}.
\end{align}
A negative rule.
\begin{align}
x^{\underline {-m}} = \frac{1}{(x+1) ... (x+m)}. &&\{\text{for } m>0\}
\end{align}
Another rule:
\begin{align}
x^{\underline {m+n}} = x^{\underline m} (x-m)^{\underline n}.
\end{align}
A complete description of the sums of falling powers.
\begin{align}
\sum\nolimits_{a}^{b}x^{\underline m}\delta x = 
\begin{cases}
\frac{k^{\underline {m+1}}}{m+1}\Big|_a^b; &\{\text{for } m \neq -1\}\\
H_x\big|_a^b.  &\{\text{for } m = -1\}
\end{cases}
\end{align}
Corresponding to $D(e^x) = e^x$:
\begin{align}
\bigtriangleup 2^x = 2^{x+1} - 2^x = 2^x.
\end{align}
One summary:
\\

\begin{tabular}{l l l l l}
\hline
$f = \sum g$ & $\bigtriangleup f = g$ & & $f = \sum g$ & $\bigtriangleup f = g$\\
\hline
$x^{\underline 0} = 1$ & $0$ & & $2^x$& $2^x$\\
$x^{\underline 1} = x$ & $1$ & & $c^x$& $(c-1)c^x$\\
$x^{\underline 2} = x(x-1)$ & $2x$ & & $c^x/(c-1)$& $c^x$\\
$x^{\underline m}$ & $mx^{\underline {m-1}}$ & & $cf$& $c\bigtriangleup f$\\
$x^{\underline {m+1}}/(m+1)$ & $x^{\underline m}$ & & $f+g$ & $\bigtriangleup f + \bigtriangleup g$\\
$H_x$ & $x^{\underline {-1}} = 1 / (x+1)$ & & $fg$& $f\bigtriangleup g + g\bigtriangleup f$\\
\hline
\end{tabular}
\\

$\bigtriangleup (u(x)v(x))$ does not have a nice form:
\begin{align}
\bigtriangleup(u(x)v(x)) &= u(x+1)v(x+1) - u(x)v(x) \\
			 &= u(x+1)v(x+1) - u(x)v(x+1) + u(x)v(x+1) - u(x)v(x)\\
			 &= u(x)\bigtriangleup v(x) + v(x+1)\bigtriangleup u(x).
\end{align}
Define 
\begin{align}
Ef(x) = f(x+1).
\end{align}
There is
\begin{align}
\bigtriangleup (uv) = u\bigtriangleup v + Ev\bigtriangleup u.
\end{align}
and 
\begin{align}
\sum u \bigtriangleup v = uv - \sum Ev \bigtriangleup u.
\end{align}
Example one.
\begin{align}
\sum x2^x \delta x = x2^x - \sum 2^{x+1} \delta x = x2^x - 2^{x+1} + C.
\end{align}
Example two.
\begin{align}
\sum xH_x \delta x &= \sum H_x \delta \frac{1}{2}x^{\underline 2} \\
		   &= \frac{x^{\underline 2}}{2}H_x - \sum \frac{1}{2}(x+1)^{\underline 2}\delta H_x \\
		   &= \frac{x^{\underline 2}}{2}H_x - \sum \frac{1}{2}(x+1)^{\underline 2}x^{\underline -1}\delta x \\
		   &= \frac{x^{\underline 2}}{2}H_x - \sum \frac{1}{2}x^{\underline 1} \delta x \\
		   &= \frac{x^{\underline 2}}{2}H_x - \frac{1}{4}x^{\underline 2} + C.
\end{align}

\subsection{INFINITE SUMS}


\end{document}
