\documentclass{article}
\usepackage[a4paper,hmargin=1.25in,vmargin=1in]{geometry}

\usepackage{listings} \usepackage[usenames,dvipsnames]{color}
\usepackage{amsmath}
\usepackage{latexsym}
\usepackage{mathtools}

\DeclarePairedDelimiter\ceil{\lceil}{\rceil}
\DeclarePairedDelimiter\floor{\lfloor}{\rfloor}

\definecolor{DarkGreen}{rgb}{0.0,0.4,0.0}
\definecolor{highlight}{RGB}{255,251,204}

\lstdefinestyle{Style1}{
language=PYTHON, 
backgroundcolor=\color{highlight}, 
basicstyle=\footnotesize\ttfamily,
breakatwhitespace=false,
breaklines=true,
captionpos=b,
commentstyle=\usefont{T1}{pcr}{m}{sl}\color{DarkGreen},
deletekeywords={},
%escapeinside={\%}, % This allows you to escape to LaTeX using the character in the bracket
firstnumber=1,
frame=single,
frameround=tttt,
keywordstyle=\color{Blue}\bf,
morekeywords={},
numbers=left,
numbersep=10pt,
numberstyle=\tiny\color{Gray},
rulecolor=\color{black},
showstringspaces=false,
showtabs=false,
stepnumber=1,
stringstyle=\color{Purple},
tabsize=2,
}

\newcommand{\insertcode}[2]{\begin{itemize}\item[]\lstinputlisting[caption=#2,label=#1,style=Style1]{#1}\end{itemize}}
%\newcommand\floor[1]{\lfloor#1\rfloor}
%\newcommand\ceil[1]{\lceil#1\rceil}

\begin{document}

\setcounter{section}{2}
\section{Integer Functions}
\subsection{FLOORS AND CEILINGS}
Define: $\ceil{x}$ is the least integer greater than or equal to $x$, and $\floor{x}$ is the greatest integer less than or equal to $x$.
Basic rules:
\begin{align}
\floor{x} &\le x; \\
\ceil{x} &\ge x.
\end{align}
The two functions are equal precisely at the integer points:
\begin{align}
\floor{x} = x \iff x \text{ is an integer} \iff \ceil{x} = x.
\end{align}
The two functions are inequal if not at the integer points:
\begin{align}
\ceil{x} - \floor{x} &= [x \text{ is not an integer}].
\end{align}
The two functinos can be converted:
\begin{align}
\ceil{-x} &= -\floor{x}; \\
\floor{-x} &= -\ceil{x}.
\end{align}
Integers can be easily removed or added in the two functions:
\begin{align}
\floor{x + n} &= \floor{x} + n; && \{\text{integer n}\} \\
\ceil{x + n} &= \ceil{x} + n. && \{\text{integer n}\}
\end{align}
For important rules:
\begin{align}
\floor{x} = n &\iff x-1 < n \le x < n+1;\\
\ceil{x} = n &\iff n-1 < x \le n < x + 1.
\end{align}
There are many situations in which floor and ceiling brackets are redundant:
\begin{align}
x < n &\iff \floor{x} < n;\\
n < x &\iff n < \ceil{x};\\
x \le n &\iff \ceil{x} \le n;\\
n \le x &\iff n \le \floor{x}.
\end{align}
Define: $\{x\} = x - \floor{x}$ is the fractional part of $x$, then $\floor{x}$ is the integer part of $x$.
A simple notation is $x = n + \theta$.
\begin{align}
\floor{x + y} = \floor{x} + \floor{y} + \floor{\{x\} + \{y\}}.
\end{align}
\subsection{FLOOR/CEILING APPLICATIONS}
Problem 1: what is the bit number to express n in binary?
\begin{align}
&2^{m-1} \le x < 2^m \iff \text{the bit number is }m; \\
&m-1 \le \lg{x} < m \\
&m = \floor{\lg{x}} + 1. && \{x > 0\}
\end{align}
To support $x=0$, another better solution is $\ceil{\lg(x+1)}$.\\
Problem 2: what is $m = \floor{\sqrt{\floor{x}}}$ when $x \ge 0$?
\begin{align}
&m \le \sqrt{\floor{x}} < m + 1; \\
&m^2 \le \floor{x} < (m+1)^2; \\
&m^2 \le x < (m+1)^2; \\
&m \le \sqrt{x} < m + 1; \\
&m = \floor{\sqrt{x}}.
\end{align}
Problem 3: what is $m = \ceil{\sqrt{\ceil{x}}}$ when $x \ge 0$?
\begin{align}
&m - 1 < \sqrt{\ceil{x}} \le m; \\
&(m-1)^2 < \ceil{x} \le m^2; \\
&(m-1)^2 < x \le m^2; \\
&m-1 < \sqrt{x} \le m; \\
&m = \ceil{\sqrt{x}}.
\end{align}
A general theorem: let $f(x)$ be any continuous, monotonically increasing function with the property that
\begin{align}
f(x) = \text{integer} \implies x = \text{integer}.
\end{align}
Then there is:
\begin{align}
\floor{f(x)} = \floor{f(\floor{x})};\\
\ceil{f(x)} = \ceil{f(\ceil{x})}.
\end{align}
A special case of the theorem:
\begin{align}
\floor*{\frac{x+m}{n}} = \floor*{\frac{\floor{x}+m}{n}};\\
\ceil*{\frac{x+m}{n}} = \ceil*{\frac{\ceil{x}+m}{n}}.
\end{align}
Problem levels: 
\textbf{level 1} prove a given statement for a number;
\textbf{level 2} prove a given statement for a set of numbers;
\textbf{level 3} prove or disprove a given statement for a set of numbers;
\textbf{level 4} find a necessary and suffcient condition that a statement is true;
\textbf{level 5} find an interesting property given a set of numbers.\\
Consider the integer inside a range:
\begin{align}
\alpha \le n < \beta \iff \ceil{\alpha} \le n < \ceil{\beta};\\
\alpha < n \le \beta \iff \floor{\alpha} < n \le \floor{\beta}.
\end{align}
Then
\begin{align}
&[\alpha, \beta) \text{ contains } \ceil{\beta} - \ceil{\alpha}   \text{ elements}; && \{\alpha \le \beta\} \\
&(\alpha, \beta] \text{ contains } \floor{\beta} - \floor{\alpha} \text{ elements}; && \{\alpha \le \beta\} \\
&(\alpha, \beta) \text{ contains } \ceil{\beta} - \floor{\alpha}-1 \text{ elements}; && \{\alpha < \beta\} \\
&[\alpha, \beta] \text{ contains } \floor{\beta} - \ceil{\alpha}+1 \text{ elements}. && \{\alpha \le \beta\} 
\end{align}
Example 1:
\begin{align}
W &= \sum_{1\le n \le 1000} [\floor{\sqrt[3]{n}} \setminus n] \\
  &= \sum_{k,n} [k = \floor{\sqrt[3]{n}}] [1 \le n \le 1000] [k \setminus n] \\
  &= \sum_{k,n,m} [k^3 \le n < (k+1)^3] [n = km] [1 \le n \le 1000] \\
  &= 1 + \sum_{k,m} [k^3 \le km < (k+1)^3] [1 \le k < 10]\\
  &= 1 + \sum_{k,m} [k^2 \le m < (k+1)^3/k] [1 \le k < 10]\\
  &= 1 + \sum_{1\le k < 10} (\ceil{(k+1)^3/k} - \ceil{k^2})\\
  &= 1 + \sum_{1\le k < 10} (3k+4) = 172.
\end{align}
General case:
\begin{align}
W &= \sum_{1\le n \le N} [\floor{\sqrt[3]{n}} \setminus n] \\
  &= \sum_{k,n} [k = \floor{\sqrt[3]{n}}] [1 \le n \le N] [k \setminus n] \\
  &= \sum_{k,n,m} [k^3 \le n < (k+1)^3] [n = km] [1 \le n \le N] \\
  &= \sum_{k,m} [k^3 \le km < (k+1)^3] [1 \le k < K] + \sum_{k,m} [K^3 \le Km \le N] \\
  &= \sum_{k,m} [k^2 \le m < (k+1)^3/k] [1 \le k < K] + \sum_{k,m} [K^2 \le m \le N/K] \\
  &= \sum_{1 \le k < K} (3k+4) + \sum_{m} [m \in [K^2, N/K]]\\
  &= (7+3K+1)(K-1)/2 + \floor{N/K} - \ceil{K^2} + 1\\
  &= \frac{1}{2}K^2 + \frac{5}{2}K -3 + \floor{N/K}. &&\{K=\floor{\sqrt[3]{N}}\}
\end{align}
Define $Spec(\alpha) = \{\floor{\alpha}, \floor{2\alpha},...\}$ then $Spec(\sqrt{2})$ and $Spec(\sqrt{2}+2)$ forms a partition of positive integers.
Define $N(\alpha,n)$ is the number of elements in $Spec(\alpha)$ that are $\le n$.
\begin{align}
N(\alpha,n) &= \sum_{k>0}[\floor{\alpha k} \le n] \\
	    &= \sum_{k>0}[\floor{\alpha k} < n + 1] \\
	    &= \sum_{k>0}[\alpha k < n + 1] \\
	    &= \sum_{k>0}[ 0 < k < (n + 1)/\alpha] \\
	    &= \ceil{(n+1)/\alpha} - 1.
\end{align}
Then $N(\sqrt{2}, n) + N(\sqrt{2}+2, n) = n$. 
And it is easy to prove that if $\alpha \neq \beta$ then $Spec(\alpha) \neq Spec(\beta)$.
\subsection{FLOOR/CEILING RECURRENCES}
Knuth numbers:
\begin{align}
K_0 &= 1; \\
K_{n+1} &= 1 + min(2K_{\floor{n/2}}, 3K_{\floor{n/3}}).
\end{align}
The Josephus problem:
\begin{align}
J(1) &= 1; \\
J(n) &=  2J(\floor{n/2}) - (-1)^n.
\end{align}
Consider the more authentic Josephus problem in which every third person is eliminated:
\begin{align}
J_3(n) = \ceil*{\frac{3}{2}J_3(\floor*{\frac{2}{3}n}) + a_n} \mod (n+1).
\end{align}
where $a_n = -2,1,-\frac{1}{2}$ when $n \mod 3 = 0,1,2$.\\
Another method: Whenever a person is passed over, we can assign a new number.
\begin{center}
\begin{tabular}{ c c c c c c c c c c }
 \hline \\
 1 & 2 & 3 & 4 & 5 & 6 & 7 & 8 & 9 & 10 \\ 
 11 & 12 & & 13 & 14 & & 15 & 16 & & 17 \\  
 18 & & & 19 & 20 & & & 21 & & 22 \\
   &   &   & 23 & 24 &  & & & & 25 \\
   &   &   &  26 & &  & & & & 27 \\
   &   &   &  28 & &  & & & &  \\
   &   &   &  29 & &  & & & &  \\
   &   &   &  30 & &  & & & &  \\
\hline
\end{tabular}
\end{center}
Any number $3k+1$ has a next value $10 + 3k + 1 - k$, and $3k + 2$ has a next value $10 + 3k + 2 - k$.
More general, there are $n$ people at first and some person has a current number $N$.
For this person his last number should be $3k+1$ or $3k+2$, and this current number is $N = n + 2k+1$ or $N = n+2k+2$.
This means 
\begin{align}
k = \floor*{\frac{N - n - 1}{2}}.
\end{align}
And his last number can be converted into
\begin{align}
3k + (N - n - 2k) = k + N - n = \floor*{\frac{N - n - 1}{2}} + N - n.
\end{align}
For the last one to be terminated, his number should be $3n$.
Use the method we can always find his last number until the number is smaller than n which is his initial number.

\insertcode{"c3/a0.py"}{Method 0}

Let $D = 3n + 1 - N$, then $D = 1$ when $N = 3n$ and $D > 2n + 1$ when $N < n$.
$D$ can also be updated as $N$:
\begin{align}
D &= 3n + 1 - N  \\
  &= 3n + 1 - (\floor*{\frac{(3n + 1 - D) - n - 1}{2}} + (3n + 1 -D) - n)  \\
  &= n + D - \floor*{\frac{2n - D}{2}}  \\
  &= D - \floor*{\frac{-D}{2}} \\
  &= D + \ceil*{\frac{D}{2}} \\
  &= \ceil*{\frac{3D}{2}}.
\end{align}

\insertcode{"c3/a1.py"}{Method 1}

More general:

\insertcode{"c3/a2.py"}{Method 2}

Write it into a recurrence:
\begin{align}
D_0^{(q)} &= 1; \\
D_n^{(q)} &= \ceil*{\frac{q}{q-1} D_{n-1}^{(q)}}.
\end{align}

\subsection{`MOD': THE BINARY OPERATION}
Define operator `mod':
\begin{align}
x \text{ mod } y = x - y \floor{x / y}. && \{y \neq 0\}
\end{align}
Based on the defination, there are some attributes:
\begin{align}
0 \le x \text{ mod } y < y; && \{ y > 0\} \\
0 \ge x \text{ mod } y > y. && \{ y < 0\}
\end{align}
To complete the defination, we can let $x \text{ mod } y = x$ when $y = 0$.\\
The `mod' operator can be used to show the fractional part of a number:
\begin{align}
x = \floor{x} + x \text{ mod } 1.
\end{align}
A similar `mumble' operator can be defined:
\begin{align}
x \text{ mumble } y = y \ceil{x / y} - x. && \{y \neq 0\}
\end{align}
The `mod' operator follows the distributive law:
\begin{align}
c(x \text{ mod } y) = (cx) \text{ mod } (cy).
\end{align}
Problem: how to partition n things into m groups as equally as possible?\\
There will be $n \text{ mod } m$ groups contains $\ceil{n/m}$ things and the rest contains $\floor{n/m}$ things.
It also can be converted into:
\begin{align}
n = \ceil*{\frac{n}{m}} + \ceil*{\frac{n-1}{m}} +  ... + \ceil*{\frac{n - m + 1}{m}}.
\end{align}
and if change n to $km + n \text{ mod } m$, the equation can be converted into:
\begin{align}
n = \floor*{\frac{n}{m}} + \floor*{\frac{n+1}{m}} +  ... + \floor*{\frac{n + m - 1}{m}}.
\end{align}
If $n = \floor{mx}$:
\begin{align}
\floor{mx} = \floor*{x} + \floor*{x + \frac{1}{m}} +  ... + \floor*{x + \frac{m - 1}{m}}.
\end{align}

\subsection{FLOOR/CEILING SUMS}
Example 1, let $a = \floor{\sqrt{n}}$:
\begin{align}
\sum_{0 \le k < n} \floor{\sqrt{k}} &= \sum_{k,m \ge 0} m[m = \floor{\sqrt{k}}][k < n] \\
				    &= \sum_{k,m \ge 0} m[k<n][m \le \sqrt{k} < m+1] \\
				    &= \sum_{k,m \ge 0} m[k<n][m^2 \le k < (m+1)^2 \\
				    &= \sum_{k,m \ge 0} m[m^2 \le k < (m+1)^2 \le n]  + \sum_{k,m \ge 0} m[m^2 \le k < n < (m+1)^2] \\
				    &= \sum_{m \ge 0} m ((m+1)^2 - m^2)[m+1 \le a] + \sum_{m \ge 0} m(a^2 \le k < n) \\
				    &= \sum_{m \ge 0} (2m^2 + m )[m+1 \le a] + a(n - a^2) \\
				    &= \sum_{m \ge 0}  (2m^{\underline{2}} + 3m^{\underline{1}}) [m < a] - a^3 + an \\
				    &= \sum\nolimits_0^a (2m^{\underline{2}} + 3m^{\underline{1}})\delta m - a^3 + an \\
				    &= \frac{2}{3}m^{\underline{3}} + \frac{3}{2}m^{\underline{2}} - a^3 + an \\
				    &= na - \frac{1}{3}a^3 - \frac{1}{2}a^2 - \frac{1}{6}a.
\end{align}
Anothe method is le $\floor{x} = \sum_j [1 \le j \le x]$:
\begin{align}
\sum_{0 \le k < n} \floor{\sqrt{k}} &= \sum_{j,k} [1 \le j \le \sqrt{k}][0 \le k \le a^2] \\
				    &= \sum_{1 \le j < a} \sum_k [j^2 \le k < a^2] \\
				    &= \sum_{1 \le j < a} (a^2 - j^2) \\
				    &= na - \frac{1}{3}a^3 - \frac{1}{2}a^2 - \frac{1}{6}a.
\end{align}
Equidistribution theorem:
\begin{align}
\lim_{x\to\infty} \frac{1}{n}\sum_{0 \le k < n} f(\{k\alpha\}) = \int_{0}^{1} f(x) dx.
\end{align}
for all irrational $\alpha$ and all functions f that are continuous almost everywhere.\\
Example 2, let $d = gcd(m,n)$:
\begin{align}
\sum_{0 \le k < m} \floor*{\frac{nk + x}{m}} &= d\floor*{\frac{x}{d}} + \frac{m-1}{2}n + \frac{d-m}{2} \\
					     &= d\floor*{\frac{x}{d}} + \frac{mn}{2} - \frac{n}{2} - \frac{m}{2} + \frac{d}{2} \\
					     &= \sum_{0 \le k < n} \floor*{\frac{mk + x}{n}}.
\end{align}

\subsection{Exercises}
\textbf{Warmups 3.1:}
Let
\begin{align}
n = 2^m + l. && \{0 \le l < 2^m\}
\end{align}
Then:
\begin{align}
m &= \ceil{\lg{(n+1)}}; \\
l &= n - 2^{\ceil{\lg{(n+1)}}}.
\end{align}

\noindent\rule{\textwidth}{0.4pt}
\textbf{Warmups 3.2:}
\begin{align}
round_{down}(x) = \floor{x + 0.5}; \\
round_{up}(x) = \ceil{x - 0.5}.
\end{align}

\noindent\rule{\textwidth}{0.4pt}
\textbf{Warmups 3.3:}
\begin{align}
\floor*{\frac{\floor{m\alpha}n}{\alpha}} = \floor*{\frac{m\alpha n - \{m\alpha\}n}{\alpha}} = mn - \ceil*{\frac{\{m\alpha\}n}{\alpha}} = mn - 1.
\end{align}

\noindent\rule{\textwidth}{0.4pt}
\textbf{Warmups 3.4:}
Pass.

\noindent\rule{\textwidth}{0.4pt}
\textbf{Warmups 3.5:}
\begin{align}
\floor{n\floor{x} + n\{x\}} = n\floor{x} \iff \floor{n\{x\}} = 0 \iff 0 \le n\{x\} < 1 \iff \{x\} < \frac{1}{n}.
\end{align}

\noindent\rule{\textwidth}{0.4pt}
\textbf{Warmups 3.6:}
\begin{align}
\floor{f(\ceil{x})} = A &\iff A \le f(\ceil{x}) < A + 1 \\
			&\iff  f^{-1}(A) \ge \ceil{x} > f^{-1}(A+1) \\
			&\iff f^{-1}(A) \ge x > f^{-1}(A+1) \\
			&\iff \floor{f(\ceil{x})} = \floor{f(x)}.
\end{align}

\noindent\rule{\textwidth}{0.4pt}
\textbf{Warmups 3.7:}
\begin{align}
X_n &= n; && \{0 \le n < m\}\\
X_n &= X_{n-m} + 1. && \{n \ge m\}&& 
\end{align}
Solution:
\begin{align}
X_n = \floor*{\frac{n}{m}} + n \text{ mod } m.
\end{align}

\noindent\rule{\textwidth}{0.4pt}
\textbf{Warmups 3.8:}
If $m$ boxes contains $ < \ceil{n/m}$ elements:
\begin{align}
n \le m (\ceil{n/m} - 1) \iff n/m + 1 \le \ceil{n/m}.
\end{align}
If $m$ boxes contains $ > \floor{n/m}$ elements:
\begin{align}
n \ge m (\floor{n/m} + 1) \iff n/m - 1 \ge \floor{n/m}.
\end{align}
These two statements contradict function 9 and 11.

\noindent\rule{\textwidth}{0.4pt}
\textbf{Warmups 3.9:}
Because
\begin{align}
\frac{m}{n} - \frac{1}{q} = \frac{m \ceil{\frac{n}{m}}-n}{nq} = \frac{n \text{ mumble } m}{nq}.
\end{align}
Then $n \text{ mumble } m$ is smaller than $m$ and $nq$ is larger than q.
This means 1) it is possible to splite a fractional number into a number series;
2) The number series $1/x_1,1/x_2,...$ has distinct numbers;
3) This is a finite number series.

\noindent\rule{\textwidth}{0.4pt}
\textbf{Basics 3.10:}
\begin{align}
\ceil*{\frac{2x + 1}{2}} - \ceil*{\frac{2x + 1}{4}} + \floor*{\frac{2x + 1}{4}} = \ceil{x + 0.5} - [x \neq 2k-0.5]. && \{k \text{ is a integer}\}
\end{align}
This means:
\begin{align}
\ceil*{\frac{2x + 1}{2}} - \ceil*{\frac{2x + 1}{4}} + \floor*{\frac{2x + 1}{4}} &= \ceil{x}; && \{x = 2k-0.5 \text{ or } \{x\}>0.5\} \\
										&= \floor{x}. && \{\text{else}\}
\end{align}

\noindent\rule{\textwidth}{0.4pt}
\textbf{Basics 3.11:}
\begin{align}
\alpha < n < \beta \iff \floor{\alpha} < n < \ceil{\beta}.
\end{align}
The number of possible n is
\begin{align}
(\ceil{\beta} - \floor{\alpha} - 1)[\ceil{\beta} > \floor{\alpha}].
\end{align}
If $\alpha = \beta = \text{integer}$ then $\ceil{\beta} = \floor{\alpha}$, this reults a wrong answer.

\noindent\rule{\textwidth}{0.4pt}
\textbf{Basics 3.12:}
Prove
\begin{align}
\ceil*{\frac{n}{m}} = \floor*{\frac{n+m-1}{m}} &\iff \ceil*{\frac{km + n\text{ mod } m}{m}} = \floor*{\frac{km + n\text{ mod } m +m-1}{m}} \\
					       &\iff k + \ceil*{\frac{n\text{ mod } m}{m}} = k + 1 + \floor*{\frac{ n\text{ mod } m -1}{m}} \\
\end{align}
If $n \text{ mod } m = 0$
\begin{align}
\ceil*{\frac{n}{m}} = \floor*{\frac{n+m-1}{m}} \iff k = k + 1 - 1.
\end{align}
Else
\begin{align}
\ceil*{\frac{n}{m}} = \floor*{\frac{n+m-1}{m}} \iff k + 1 = k + 1 - 0.
\end{align}

\noindent\rule{\textwidth}{0.4pt}
\textbf{Basics 3.13:}
Because
\begin{align}
N(\alpha,n) + N(\beta,n) &= N(\frac{\beta}{\beta-1},n) + N(\beta,n) \\
			 &= n+1 + \ceil*{\frac{n+1}{\beta}} - \floor*{\frac{n+1}{\beta}} - 2 \\
			 &= n.
\end{align}
And it is easy to prove that if $\alpha \neq \beta$ then $Spec(\alpha) \neq Spec(\beta)$.

\noindent\rule{\textwidth}{0.4pt}
\textbf{Basics 3.14:}
\begin{align}
(x \text{ mod } ny) \text{ mod } y &= x - \floor*{\frac{x}{ny}}ny - \floor*{\frac{x - \floor*{\frac{x}{ny}}ny }{y}}y \\
				   &= x - y(\floor*{\frac{x}{ny}}n - \floor*{\frac{x}{y} + \floor*{\frac{x}{ny}}n}) \\
				   &= x - \floor*{\frac{x}{y}}y = x \text{ mod } y.
\end{align}

\noindent\rule{\textwidth}{0.4pt}
\textbf{Basics 3.15:}
\begin{align}
\ceil{mx} = \ceil*{x} + \ceil*{x - \frac{1}{m}} +  ... + \ceil*{x - \frac{m - 1}{m}}.
\end{align}

\noindent\rule{\textwidth}{0.4pt}
\textbf{Basics 3.16:}
Prove
\begin{align}
n \text{ mod } 2 = (1 - (-1)^n)/2.
\end{align}
It is true in both even and odd cases.\\
Solve
\begin{align}
n \text{ mod } 3 = a + bw^n + cw^{2n}. && \{w = (-1 + i\sqrt{3})/2\}
\end{align}
Try $n = 0,1,2$:
\begin{align}
a + b + c &= 0; \\
a + bw + cw^2 &= a + bw + c(-1-w) = 1; \\
a + bw^2 + cw^4 &= a + b(-1-w) + cw = 2.
\end{align}
Solution is $a = 1$, $b =(-1-w)/(1+2w) = (w-1)/3$, $c = -1-b = -(w+2)/3$.\\
To prove this is easy, because $w^3 = 1$.

\noindent\rule{\textwidth}{0.4pt}
\textbf{Basics 3.17:}
\begin{align}
\sum_j \sum_k &[0 \le k < m][1 \le j \le x + k/m] \\
	&= \sum_j \sum_k [0 \le k < m][1 \le j \le \ceil{x}][j \le x + k/m] \\
	&= \sum_j \sum_k [0 \le k < m][1 \le j \le \ceil{x}][ k \ge m(j-x)] \\
	&= \sum_{1 \le j \le \ceil{x}} \sum_k [0 \le k < m] - \sum_{j=\ceil{x}} \sum_k [0 \le k < m(j-x)] \\
	&= m\ceil{x} - \ceil{m(\ceil{x}-x)} \\
	&= \floor{mx}.
\end{align}

\noindent\rule{\textwidth}{0.4pt}
\textbf{Basics 3.18:}
pass.

\noindent\rule{\textwidth}{0.4pt}
\textbf{Homework exercises 3.19:}
To let $f(x) = \floor{log_b (x)} = \floor{log_b (\floor{x})}$:
\begin{align}
f(x) = \text{integer} \implies x = \text{integer}.
\end{align}
Then $b$ should be an integer.

\noindent\rule{\textwidth}{0.4pt}
\textbf{Homework exercises 3.20:}
\begin{align}
\sum_k [\alpha \le kx \le \beta] xk &= x\sum_k [\alpha/x \le k \le \beta/x] k \\
				    &= x\sum_k [\ceil{\alpha/x} \le k \le \floor{\beta/x}] k \\
				    &= \frac{1}{2}x(\floor{\beta/x}\floor{\beta/x} + \floor{\beta/x} - \ceil{\alpha/x}\ceil{\alpha/x} + \ceil{\alpha/x}).
\end{align}

\noindent\rule{\textwidth}{0.4pt}
\textbf{Homework exercises 3.21:}
Small cases $1, 16, 128, 1024, ...$ show that for a number which has $k$ digits in decimal notation, there always exactly one number $2^m$ has leading 1.
This is true because for any number $n$:
\begin{align}
log_{10} (n) < \floor{log_{10}(n)+1} \iff 2n < n + 10^{\floor{log_{10}(n)+1}}.
\end{align}
Then solution is $\floor{M log_{10}2} + 1$.

\noindent\rule{\textwidth}{0.4pt}
\textbf{Homework exercises 3.22:}
\begin{align}
S_n = \sum_{k \ge 1} \floor*{\frac{n}{2^k} + \frac{1}{2}}.
\end{align}
Small cases show that $S_n = n$.
Let $n = m_k 2^k + d$
\begin{align}
\floor*{\frac{n}{2^k} + \frac{1}{2}} &\neq \floor*{\frac{n-1}{2^k} + \frac{1}{2}} \iff \\
				     &m_k + \floor*{\frac{d}{2^k} + \frac{1}{2}} \neq m_k + \floor*{\frac{d-1}{2^k} + \frac{1}{2}} \iff \\
				     &d = 2^{k-1}.
\end{align}
Consider $S_n$ and $S_{n-1}$, there always one and only one k to make $n \text{ mod } (2^k) = 2^{k-1}$.
Let $n = 2^{k'} q$ where $q$ is a odd, here $k'$ is the one.
Then $S_n = S_{n-1} + 1 = n$.

\begin{align}
T_n = \sum_{k \ge 1} 2^k \floor*{\frac{n}{2^k} + \frac{1}{2}}^2.
\end{align}
From the observation above, let $n = 2^{k'} q$ where $q$ is a odd.
The different terms between $T_n$ and $T_{n-1}$ is $\floor*{\frac{2^{k'} q}{2^{k'+1}} + \frac{1}{2}}$ and $\floor*{\frac{2^{k'} q-1}{2^{k'+1}} + \frac{1}{2}}$.
Then:

\begin{align}
T_n - T_{n-1} &= 2^{k'+1}(\floor*{\frac{2^{k'} q}{2^{k'+1}} + \frac{1}{2}}^2 - \floor*{\frac{2^{k'} q-1}{2^{k'+1}} + \frac{1}{2}}^2) \\
	      &= 2^{k'+1}(\floor*{\frac{q}{2} + \frac{1}{2}}^2 - \floor*{\frac{q}{2} - \frac{1}{2^{k'+1}} + \frac{1}{2}}^2) \\
	      &= 2^{k'+1}(\floor*{\frac{q}{2} + \frac{1}{2}}^2 - \floor*{\frac{q}{2}}^2)\\
	      &= 2^{k'+1}(\floor*{\frac{q}{2} + \frac{1}{2}} - \floor*{\frac{q}{2}})(\floor*{\frac{q}{2} + \frac{1}{2}} + \floor*{\frac{q}{2}})\\
	      &= 2^{k'+1}q = 2n.
\end{align}
Then $T_n = n(n+1)$.

\noindent\rule{\textwidth}{0.4pt}
\textbf{Homework exercises 3.23:}
The nth elements should be $X_n = m$
\begin{align}
\frac{1}{2}m(m-1) &< n \le \frac{1}{2}m(m+1) \iff  \\
		  &m(m-1) < 2n \le m(m+1) \iff  \\
		  &m(m-1) + \frac{1}{4} < 2n < m(m+1) + \frac{1}{4} \iff \\
		  &m - \frac{1}{2} < \sqrt{2n} < m + \frac{1}{2} \\
		  &m < \sqrt{2n} + \frac{1}{2} \\
		  &m = \floor*{\sqrt{2n} + \frac{1}{2}}.
\end{align}

\noindent\rule{\textwidth}{0.4pt}
\textbf{Homework exercises 3.24:}
$Spec(\alpha/(\alpha + 1)) = Spec(\alpha) + n + 1$.

\noindent\rule{\textwidth}{0.4pt}
\textbf{Homework exercises 3.25:}
Can prove $K_n > n$ with induction.\\
The initial cases are true:
\begin{align}
K_0 &= 1; \\
K_1 &= 3.
\end{align}
Assume that $K_i > i$ when $i\le n$, then:
\begin{align}
K_{n+1} &= 1 + min(2K_{\floor{n/2}}, 3K_{\floor{n/3}}) \\
	&\ge 1 + min(2(\floor{n/2}+1),3(\floor{n/3}+1)) \\
	&\ge 1 + min(n+1, n+1) \\
	&> n+1.
\end{align}
Proved.

\noindent\rule{\textwidth}{0.4pt}
\textbf{Homework exercises 3.26:}
For the left side:
\begin{align}
D_n^{(q)} &= \ceil*{\frac{q}{q-1} D_{n-1}^{(q)}} \iff\\
	  & \frac{q}{q-1} D_{n-1}^{(q)} \le D_n^{(q)} \\
	  & (\frac{q}{q-1})^n D_{0}^{(q)} \le D_n^{(q)} \\
	  & (\frac{q}{q-1})^n  \le D_n^{(q)} \
\end{align}
For the right side, can prove with induction:
\begin{align}
D_n^{(q)} \le q(\frac{q}{q-1})^n - q + 1. &&\{\text{this is wired but can be guessed from the initial case.}\}
\end{align}
The initial case is true:
\begin{align}
D_0^{(q)}  = 1 \ge q - q + 1.
\end{align}
Assume that it is true when $i \le n$, then:
\begin{align}
D_{n+1}^{(q)} &= \ceil*{\frac{q}{q-1} D_{n-1}^{(q)}} \\
	      &\le \ceil*{q(\frac{q}{q-1})^{n+1} - q} \\
	      &\le q(\frac{q}{q-1})^{n+1} - q + 1.
\end{align}
Proved.

\noindent\rule{\textwidth}{0.4pt}
\textbf{Homework exercises 3.27:}
Let $D_n^{(3)} = 2^m b - a$ where $a = 0\text{ or }1$ and $b$ is odd, then:
\begin{align}
D_{n+1}^{(3)} &= \ceil*{\frac{3}{2} (2^m b - a)} = 3 * 2^{m-1} b - \floor*{\frac{3}{2}a} = 3 * 2^{m-1} b  - a; \\
D_{m+n}^{(3)} &= 3^m b - a.
\end{align}
This formula shows another way to generate the next number in the $D_n^{(3)}$.
This way does not generate the next number one by one which means the formula only generates a subset.\\
Let the seed number is $D_0^{(3)} = 2^1 - 1 = 1$, then $a=1$.
The numbers generated from the seed contians infinite numbers and these numbers are arranged in the order: odd, even, odd, even, ...

\noindent\rule{\textwidth}{0.4pt}
\textbf{Homework exercises 3.28:}
Not solved yet.\\
From an initial number $a_n = m^2$, all numbers can be generated:
\begin{align}
a_{n + 2k + 1} &= (m+k)^2 + m - k; && \{0 \le k \le m \} \\
a_{n + 2k + 2} &= (m+k)^2 + 2m; && \{0 \le k \le m \} \\
a_{n + 2m + 1} &= 4m^2.
\end{align}
Then $a_{n + 2m + 1}$ is the next initial number.

\noindent\rule{\textwidth}{0.4pt}
\textbf{Homework exercises 3.29:}
pass.

\noindent\rule{\textwidth}{0.4pt}
\textbf{Homework exercises 3.30:}
Small cases show that 
\begin{align}
X_n = \alpha^{2^n} + \alpha^{-2^n}.
\end{align}
It is easy to prove this with the induction.\\
$\alpha^{-2^n} < 1$ because $\alpha > 1$ which means the integer $X_n$ should be $\ceil*{\alpha^{2^n}}$.

\noindent\rule{\textwidth}{0.4pt}
\textbf{Homework exercises 3.31:}
A really smart solution:
\begin{align}
\floor{x} + \floor{y} + \floor{x + y} &= \floor{x + \floor{y}} + \floor{x + y} \\
				      &\le \floor*{x + \frac{1}{2}\floor{2y}} + \floor*{x + \frac{1}{2}\floor{2y} + \frac{1}{2}} \\
				      &= \floor*{2x + \floor{2y}} = \floor{2x} + \floor{2y}.
\end{align}

\noindent\rule{\textwidth}{0.4pt}
\textbf{Homework exercises 3.32:}
\begin{align}
f(x) = \sum_k 2^k ||\frac{x}{2^k}||^2. && \{||x|| = \min{x - \floor{x},\ceil{x}-x}\}
\end{align}
Firstly, consider some basic rules.\\
$f(x) = f(-x)$ because $||x|| = ||-x||$.
Also:
\begin{align}
f(2x) = \sum_k 2^k ||\frac{2x}{2^k}||^2 = 2\sum_k 2^{k-1} ||\frac{x}{2^{k-1}}||^2 = 2f(x).
\end{align}
Secondly, consider $x$ in $0 \le x < 1$.\\
Let $f(x) = l(x) + r(x)$, where $l(x)$ is the sum when $k\le0$ and $r(x)$ is the sum when $k > 0$.
\begin{align}
l(x+1) = \sum_{k \le 0} 2^k ||\frac{x+1}{2^k}||^2 = \sum_{k \le 0} 2^k ||\frac{x}{2^k} + 2^{-k}||^2 = \sum_{k \le 0} 2^k ||\frac{x}{2^k}||^2 = l(x).
\end{align}
Because $||x|| \le 0.5$ then:
\begin{align}
l(x) \le \sum_{k \le 0} 2^k (\frac{1}{2})^2 = \frac{1}{2}.
\end{align}
Then $r(x+1) = r(x) + 1$ because:
\begin{align}
r(x) &= \frac{x^2}{2} + \frac{x^2}{4} + \frac{x^2}{8} + ... = x^2. \\
r(x+1) &= 2||\frac{x+1}{2}||^2 + \frac{x^2}{4} + \frac{x^2}{8} + ... \\
       &= \frac{(x-1)^2}{2} + \frac{(x+1)^2}{4} + \frac{(x+1)^2}{8} + ... \\
       &= \frac{(x-1)^2}{2} + \frac{(x+1)^2}{2} = x^2 + 1.
\end{align}
Then $f(x+1) = l(x+1) + r(x+1) = l(x) + r(x)+1 = f(x)+ 1$ when $0\le x < 1$.
It also shows that $f(x+n) = f(x) + n$ and $f(0) = 0$ means $f(n) = n$.
Finally small cases show that $f(x) = |x|$, try to prove that when $0 \le x < 1$:
\begin{align}
f(x) &= 2^{-m}f(2^mx) \\
     &= 2^{-m} (f(\floor{2^mx} + \{2^mx\}) \\
     &= 2^{-m}\floor{2^mx} + 2^{-m}f(\{2^mx\}). &&\{m \text{ is any integer}\}
\end{align}
Because 
\begin{align}
f(\{2^mx\}) = l(\{2^mx\}) + r(\{2^mx\}) = l(\{2^mx\}) + (\{2^mx\})^2 \le 0.5 + 1 = 1.5.
\end{align}
Then 
\begin{align}
|f(x) - x| &= |2^{-m}\floor{2^mx} + 2^{-m}f(\{2^mx\}) - x| \\
	   &= |2^{-m}\floor{2^mx} - x| + 2^{-m}f(\{2^mx\})  \\
	   &\le |2^{-m}\floor{2^mx} - x| + 2^{-m}\frac{3}{2}  \\
	   &= |2^{-m}(\floor{2^mx} - 2^mx)| + 2^{-m}\frac{3}{2}  \\
	   &\le 2^{-m} \frac{5}{2}.
\end{align}
Because $m$ could be any integer, then there is $|f(x) - x| = 0$ which means $f(x) = |x|$ when $0\le x < 1$.
In summary, because $f(x) = f(-x)$ and $f(x) +n = f(x+n)$, then $f(x) = |x|$ for all real $x$. 

\noindent\rule{\textwidth}{0.4pt}
\textbf{Exam problems 3.33:}
a.Split the area into for parts, and consider the top right square.
Because $r$ is a fractional number, corners are not crossed.
The circle edge could be treated as a path go from the top to the right.
Because no left or up steps, then the step number is $r + r = 2r = 2n-1$.
Then the number of cells of the board contain a segment of the circle is $4(2n-1) = 8n-4$. \\
b.This is the Guass's Circle Problem, and I have no idea why:
\begin{align}
f(n,k) = \floor{\sqrt{r^2 - k^2}}.
\end{align}

\noindent\rule{\textwidth}{0.4pt}
\textbf{Exam problems 3.34:}
a.Small cases $0,1,2,2,3,3,3,3,4,4,4,4,4,4,4,4,5, ...$ show that it is a good idea to estimate $f(n)$ when n reached the last number in its level.
Let $m = \ceil{\lg{n}}$, and there will be $2^m$ numbers if the last level is full.
\begin{align}
f(n) + (2^m - n)m &= \sum_{k=1}^{2^m} \ceil{\lg{k}} \\
		  &= \sum_{j,k} j[j = \ceil{\lg{k}}][1 \le k \le 2^m] \\
		  &= \sum_{j,k} j[2^{j-1} < k \le 2^j][1 \le j \le m] \\
		  &= \sum_{1 \le j \le m} j2^{j-1} = m2^m - 2^m + 1.
\end{align}
Then $f(n) = mn + 2^m - 1$.\\
b.When $n = 2k$:
\begin{align}
f(k) &= \ceil{\lg{k}}k + 2^{\ceil{\lg{k}}} - 1;\\
f(2k) &= \ceil{\lg{2k}}2k + 2^{\ceil{\lg{2k}}} - 1 \\
      &= \ceil{\lg{k}+1}2k + 2^{\ceil{\lg{k}+1}} - 1 \\
      &= 2\ceil{\lg{k}}k + 2k + 2*2^{\ceil{\lg{k}}} - 1;\\
f(2k) &= 2f(k) + 2k - 1;\\
f(n)  &= n - 1 + f(\ceil{n/2}) + f(\floor{n/2}).
\end{align}
When $k = 2^m + 1$ and $n = 2k -1$.
Then $m = \ceil{\lg{k-1}}$ and $m + 1 = \ceil{\lg{k}}$:
\begin{align}
f(\floor*{\frac{2k-1}{2}}) &= f(k-1) = m(k-1)-2^m +1;\\
f(\ceil*{\frac{2k-1}{2}}) &= f(k) = (m+1)k-2*2^m +1;\\
f(2k-1) &= 2mk-m+4k-4*2^m-1.
\end{align}
Then 
\begin{align}
f(2k-1) = f(\floor*{\frac{2k-1}{2}}) + f(\ceil*{\frac{2k-1}{2}}) + 2k-1 -1.
\end{align}
When $k \neq 2^m + 1$ and $n = 2k -1$.
Then $m = \ceil{\lg{k-1}} = \ceil{\lg{k}}$:
\begin{align}
f(\floor*{\frac{2k-1}{2}}) &= m(k-1)-2^m + 1;\\
f(\ceil*{\frac{2k-1}{2}}) &= mk - 2^m + 1;\\
f(2k-1) &= 2mk-m+2k-2*2^m.
\end{align}
Then 
\begin{align}
f(2k-1) = f(\floor*{\frac{2k-1}{2}}) + f(\ceil*{\frac{2k-1}{2}}) + 2k-1 -1.
\end{align}
Proved.

\noindent\rule{\textwidth}{0.4pt}
\textbf{Exam problems 3.35:}
\begin{align}
\floor{(n+1)^2 n! e} &= \floor*{(n+1)^2 n! \sum_{k=0}^{\infty}\frac{1}{k!} } \\
		     &= \floor*{(\frac{n!}{0!} + \frac{n!}{1!} + ... + \frac{n!}{n!} + \frac{n!}{(n+1)!} + ... + \frac{n!}{\infty!})(n+1)^2}\\
		     &= \floor*{(n+1)^2 n ((n-1)! + ... + 1) + (n+1)^2 + (n+1) + \frac{n+1}{n+2} + ... + \frac{n!(n+1)^2}{\infty!}} \\
\end{align}
Add the mod operator:
\begin{align}
\floor{(n+1)^2 n! e} \text{ mod } n &= \floor*{(n+1)^2 + (n+1) + \frac{n+1}{n+2} + ... + \frac{n!(n+1)^2}{\infty!}} \text{ mod } n \\
				    &= \floor*{2 + \frac{n+1}{n+2} + ... + \frac{n!(n+1)^2}{\infty!}} \text{ mod } n
\end{align}
Because 
\begin{align}
\frac{n+1}{n+2} + ... + \frac{n!(n+1)^2}{\infty!} &= \frac{n+1}{n+2}(1 + \frac{1}{n+3} + \frac{1}{(n+3)(n+4)} + ... ) \\
						  &< \frac{n+1}{n+2}(1 + \frac{1}{n+3} + \frac{1}{(n+3)(n+3)} + ...) \\
						  &= \frac{(n+1)(n+3)}{(n+2)^2} < 1
\end{align}
Then result is $2 \text{ mod } n$.

\noindent\rule{\textwidth}{0.4pt}
\textbf{Exam problems 3.36:}
\begin{align}
\sum_{1 < k < 2^{2^n}}{\frac{1}{2^{\floor{\lg k}}4^{\floor{\lg \lg k}}}} &= \sum_{k,l,m} 2^{-l} 4^{-m} [m = \floor{\lg l}][l = \floor{\lg k}][1 < k < 2^{2^n}] \\
									 &= \sum_{k,l,m} 2^{-l} 4^{-m} [2^m \le l < 2^{m+1}][2^l \le k < 2^{l+1}][0 \le m < n] \\
									 &= \sum_{l,m} 4^{-m}[2^m \le l < 2^{m+1}][0 \le m < n] \\
									 &= \sum_{m} 2^{-m}[0 \le m < n] \\
									 &= 1 + \frac{1}{2} + ... + \frac{1}{2^{n-1}} = 2(1 - 2^{-n}).
\end{align}


\noindent\rule{\textwidth}{0.4pt}
\textbf{Exam problems 3.37:}

\noindent\rule{\textwidth}{0.4pt}
\textbf{Exam problems 3.38:}

\noindent\rule{\textwidth}{0.4pt}
\textbf{Exam problems 3.39:}

\noindent\rule{\textwidth}{0.4pt}
\textbf{Exam problems 3.40:}

\end{document}
